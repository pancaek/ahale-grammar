
\chapter{Introduction}
\section{Conventions}
In this document, italic text will be used for native \langname\ words, unless enclosed in angle brackets (\nativetext{ahale} or \orthotext{ahale}, but never \orthotext{\nativetext{ahale}}).
\aside{This aside is from an internal perspective. It may discuss things such as word choice, or the cultural significance of phrases.}
\aside{\fleuron\ This aside is from an external perspective. It may discuss design choices or other such self-imposed challenges.}
The former is found mostly in running prose, and the latter in places where a word is being used as a more self-contained example, or simply if orthography is of particular interest.

\phomtext{forward slashes} are used for phonemic transcriptions, while \phontext{square brackets} are used for phonetic transcriptions.

When words are exemplified within prose, they are often followed by a short translation. These will be written surrounded by \transtext{single quotes}.

``Double quotes'' are used for non-standard, ironic, or otherwise deviant use of terms.

\subsection{Document Structure}
Most of the content contained within this grammar will be found in the main prose. However, footnotes along with margin notes will be used for tangential or, in the case of footnotes, explanatory details about particular terms.

Margin notes will be reserved for explanation of decisions made in the various translations, or in the usage details of a specific construction.
In the case of notes from an external perspective, being preceded by a fleuron (\fleuron).

\subsection{Glossing}
Glosses are generally constructed as follows:

\section{History}
\subsection{Internal}
\subsection{External}