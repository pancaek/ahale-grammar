\chapter{Introduction}
\langname\ is primarily based on things I find neat. In particular, I was inspired to essentially rewrite an old conlang in its entirety following my reading of \cite{Otanes72}. \langname\ isn't really aposteriori, or even attempting any proper histling, I just borrow words from Tagalog on occassion. The actual tenets of mine with \langname\ are prety much:

\begin{itemize}
  \item Use a small phono (and one I find cute)
  \item Minimal case marking
  \item Some amount of nonconcatinative morphology
  \item Very flexible word order
  \item Morphosyntactic alignment other than nominative--accusative, because I've gotten bored of it
  \item To make something I could theoretically speak myself, if I put the time in to learn
  \item Have something that looks pretty written down (romanized or otherwise), in case I ever want to use it in a novel or something
\end{itemize}

I've attempted to make it naturalistic, but overall the aesthetic goals outweigh strict naturalism. This is mostly the case with the lexicon, since I mostly can't be bothered to check every single entry on CLICS.

\section{Overview}