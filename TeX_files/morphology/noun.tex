\chapter{Nominal}\label{ch:morpho-nom}
\section{Inflectional Morphology}\label{sec:morpho-nom-inf}
Inflectional morphology is very limited for nouns. The most marked nouns only inflect for case marking and plurality. However, depending on the context, plurality is optional. Depending on the constructions used, inflection of the noun may be even ungrammatical, where in other situations the same inflection would be entirely sensible.

\langname 's case system is incredibly small. it consists solely of the cases used for morphosyntactic purposes; the ergative and absolutive cases. The ergative case is marked with reduplication of the initial syllable, while the absolutive case remains unmarked. This becomes slightly less transparent when this interacts with phonotactics, and when this reduplication triggers allophony.