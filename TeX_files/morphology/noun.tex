\chapter{Nominal}
\section{Alignment}

\langname\ nouns are quite analytic. A typical noun consists of a stem, plus an affix denoting case, and optional plural marking. This may be either ergative or absolutive, though the absolutive is unmarked. The following points give an indication of when each case should be used.
\begin{itemize}
  \item The agent of a transitive verb (A) is marked with ergative case
  \item The core argument of an intransitive verb (S) and the patient of a transitive verb (P) are both marked with absolutive case.
\end{itemize}

\pex<alignment>
\a<itrns>
\begingl
\glpreamble hawi keke
\pronounced{ˈɣa.wi ˈke.kə}\endpreamble
Ø-hawi[\textsc{abs-}rabbit]
keke[\textsc{npst.ipfv-}eat]
\glft `The rabbit is eating'
\endgl

\a<trns>
\begingl
\glpreamble hahawi keke ɸumau
\pronounced{ˈɣa.ɣa.wi ˈke.kə ˈɸu.mau}\endpreamble
ha\textasciitilde hawi[\textsc{erg\textasciitilde}rabbit]
keke[\textsc{npst.ipfv-}eat]
ɸumau[\textsc{abs,}grass]
\glft `The rabbit is eating the grass'
\endgl
\xe

Note that the stressed allophone of /h/ remains when reduplicated, this will be further explored in the following sections.

Notice that ``rabbit'' is declined in a different case for these two similar sentences. The ergative is marked through reduplication of the first syllable.

\section{Plurality}

Plurals are formed with an affix \langword{me-}. To illustrate its use we can revisit Example \getfullref{alignment.trns}. `\langword{hahawi keke ɸumau}'. If we want to pluralize \langword{hahawi,} we may expect \langword{mehahawi.} This is not the case however. \langword{-me} is inserted between the root and the reduplicated ergative marking. The reduplicated segment is not changed though, so the correct plural is \langword{hamehawi.} \langword{me-} is most accurately described as an interfix.

\begin{paracol}{2}
\pex<vowel-initial>
\a
\begingl
\glpreamble ana
\pronounced{ˈa.na}\endpreamble
Ø-ana[\textsc{abs-}eye]
\endgl
\switchcolumn

\a<epenthesis>
\begingl
\glpreamble aʔana
\pronounced{ˈa.ʔa.na}\endpreamble
aʔ\textasciitilde ana[\textsc{erg\textasciitilde }eye]
\endgl
\switchcolumn

\a
\begingl
\glpreamble meana
\pronounced{ˈme.a.na}\endpreamble
<me>Ø-ana[\textsc{<pl>abs-}eye]
\endgl
\switchcolumn

\a
\begingl
\glpreamble ameana
\pronounced{ˈa.mə.a.na}\endpreamble
a<me>ana[\textsc{erg<pl>\textasciitilde}eye]
\endgl
\xe
\end{paracol}
\begin{paracol}{2}
\pex<consonant-initial>
\a
\begingl
\glpreamble hawi
\pronounced{ˈɣa.wi}\endpreamble
Ø-hawi[\textsc{abs-}rabbit]
\endgl
\switchcolumn

\a
\begingl
\glpreamble hahawi
\pronounced{ˈɣa.ɣa.wi}\endpreamble
ha\textasciitilde hawi[\textsc{erg\textasciitilde }rabbit]
\endgl
\switchcolumn

\a
\begingl
\glpreamble mehawi
\pronounced{ˈme.ha.wi}\endpreamble
<me>Ø-hawi[\textsc{<pl>abs-}rabbit]
\endgl
\switchcolumn

\a
\begingl
\glpreamble hamehawi
\pronounced{ˈɣa.mə.ɣa.wi}\endpreamble
ha<me>hawi[\textsc{erg<pl>\textasciitilde}rabbit]
\endgl
\xe
\end{paracol}

Note how many forms of \langword{hawi} maintain [ɣ] even in unstressed envirenments. This is because the reduplication causes it to be preserved.

In Example \getfullref{vowel-initial.epenthesis}, an epenthetic /ʔ/ has been inserted. This is done because of restrictions surrounding diphthongs. The full explanation can be found in \Sref{sec:phonotactics}.

For stems beginning with a syllable containing a diphthong, the reduplication surfaces a bit differently:

\begin{paracol}{2}
\pex
\a
\begingl
\glpreamble auna
\pronounced{ˈau.na}\endpreamble
Ø-auna[\textsc{abs-}moon]
\endgl
\switchcolumn

\a
\begingl
\glpreamble aʔauna
\pronounced{ˈa.ʔo.na}\endpreamble
aʔ\textasciitilde auna[\textsc{erg\textasciitilde }moon]
\endgl
\switchcolumn

\a
\begingl
\glpreamble meauna
\pronounced{ˈme.au.na}\endpreamble
<me>Ø-auna[\textsc{<pl>abs-}moon]
\endgl
\switchcolumn

\a
\begingl
\glpreamble ameauna
\pronounced{ˈa.mə.au.na}\endpreamble
a<me>auna[\textsc{erg<pl>\textasciitilde}moon]
\endgl
\xe
\end{paracol}

%TODO: Write a CVV-initial inflection example
