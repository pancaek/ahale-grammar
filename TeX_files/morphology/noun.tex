\chapter{Nominal}\label{ch:morpho-nom}
\section{Inflectional Morphology}\label{sec:morpho-nom-inf}
Inflectional morphology is very limited for nouns. The most marked nouns only inflect for case marking and plurality. However, certain constructions call for plural marking to be absent altogether, further reducing the variance in word shape.

\subsection{Case Marking}\label{sec:case-marking}
\langname 's case system is incredibly small. It consists solely of the cases used for morphosyntactic purposes --- the ergative and absolutive cases. The ergative case is marked with initial syllable reduplication, while the absolutive case remains unmarked. This becomes slightly less transparent when the reduplication triggers allophony and interacts with phonotactics.

\aside{\fleuron\ While phonological form is not the main focus of this section, the variation that can occur makes it useful. It \textit{is} true that much of the prose simply reiterates what is found in the pronunciation information found in the gloss; however, I consider this approach more useful for grasping the reasoning and the triggers behind these phonological changes.}

With \notabletext{CV-initial forms,} this is quite straightforward:

\begin{subexamples}
  \baarucols{2}
  \ex
    \preamble pane
    \pronunciation ˈpa.nə
    \gloss
      pane & person[ABS] \\
  \ex
    \preamble papane
    \pronunciation pa.ˈpa.nə
    \gloss
      pa\allo pane & ERG\allo person \\
\end{subexamples}

\notabletext{V-initial forms} are slightly different, as adjacent vowels cannot be identical. An epenthetic \phomtext{ʔ} is inserted to avoid this.

\begin{subexamples}
  \baarucols{2}
  \ex
    \preamble ana
    \pronunciation ˈa.na
    \gloss
      ana & eye[ABS] \\
  \ex
    \preamble a'ana
    \pronunciation aʔ.ˈa.na
    \gloss
      aʔ\allo ana & ERG\allo eye \\
\end{subexamples}

\begin{subexamples}
  \baarucols{2}
  \ex
    \preamble ele
    \pronunciation ə.ˈle
    \gloss
      ele & heaven[ABS] \\
  \ex
    \preamble e'ele
    \pronunciation əʔ.ə.ˈle
    \gloss
      eʔ\allo ele & ERG\allo heaven \\
\end{subexamples}

\phomtext{ə}-initial forms (such as the one above) are particularly interesting in that they most clearly show an interaction between reduplication and allophony. This form of reduplication can be analyzed as the morpheme \morphtext{V,}. As a result, \phomtext{əʔ.ə.lə} is phonetically \phontext{əʔ.ə.ˈle} rather than \phontext{əʔ.ˈe.lə}.

\notabletext{CVV-initial forms} undergo a slightly different process. The reduplication in this situation does not even apply to the entire syllable, but rather, only to the CV segment.

\begin{subexamples}
  \baarucols{2}
  \ex
    \preamble keuha
    \pronunciation ˈko.ha
    \gloss
      keuha & leaf[ABS] \\
  \ex
    \preamble kekeuha
    \pronunciation kə.ˈko.ha
    \gloss
      ke\allo keuha & ERG\allo leaf \\
\end{subexamples}

Consequently for LSA, the reduplicated vowel may be different than its phonetic realization in the stem due to an underlying diphthong.

\notabletext{VV-initial forms} work similarly to CVV forms, with the addition of an epenthetic \phomtext{ʔ} at the morpheme boundary:

\begin{subexamples}
  \baarucols{2}
  \ex
    \preamble auna
    \pronunciation ˈau.na
    \gloss
      auna & moon[ABS] \\
  \ex
    \preamble a'auna
    \pronunciation aʔ.ˈo.na
    \gloss
      aʔ\allo auna & ERG\allo moon \\
\end{subexamples}

As illustrated by the previous set of examples, this also triggers mono\-phthongization of the stem's initial syllable.

\subsection{Pluralization}\label{sec:pluralization}
Pluralization is very regularly marked through the infixation of \morphtext{me} between the reduplicated syllable of an ergative form and the stem. In absolutive forms, this gives \morphtext{me} the appearance of a prefix.

Let's revisit some of the previous examples with plural forms:

\begin{subexamples}
  \baarucols{2}
  \ex
    \preamble mepane
    \pronunciation mə.ˈpa.nə
    \gloss
      <me>pane & <PL>person[ABS] \\
  \ex
    \preamble pamepane
    \pronunciation pa.ˈme.pa.nə
    \gloss
      pa\allo <me>pane & ERG\allo <PL>person \\
\end{subexamples}

\begin{subexamples}
  \baarucols{2}
  \ex
    \preamble meana
    \pronunciation mə.ˈa.na
    \gloss
      <me>ana & <PL>eye[ABS] \\
  \ex
    \preamble ameana
    \pronunciation a.ˈme.a.na
    \gloss
      a\allo <me>ana & ERG\allo <PL>eye \\
\end{subexamples}


\begin{subexamples}
  \label{ex:double-schwa}
  \baarucols{2}
  \ex
  \preamble meele
  \pronunciation ˈme.lə
  \gloss
    <me>ele & <PL>heaven[ABS] \\
  \ex
    \label{ex:double-schwa-erg}
    \preamble emeele
    \pronunciation ə.ˈme.lə
    \gloss
      e\allo <me>ele & ERG\allo <PL>heaven \\
\end{subexamples}

\aside{Loanwords from other languages are commonly respelled with \orthotext{ee} for \phontext{e}. Older generations generally dislike this emerging tendency, as it reinforces foreign loans.}

\begin{subexamples}
  \baarucols{2}
  \ex
    \preamble mekeuha
    \pronunciation mə.ˈko.ha
    \gloss
      <me>keuha & <PL>leaf[ABS] \\
  \ex
    \preamble kemekeuha
    \pronunciation kə.ˈme.ko.ha
    \gloss
      ke\allo ke<me>uha & ERG\allo <PL>leaf \\
\end{subexamples}

\begin{subexamples}
  \baarucols{2}
  \ex
    \preamble meauna
    \pronunciation mə.ˈau.na
    \gloss
      <me>auna & <PL>moon[ABS] \\
  \ex
    \preamble ameauna
    \pronunciation a.ˈme.o.na
    \gloss
      a\allo <me>auna & ERG\allo <PL>moon \\
    \end{subexamples}

\baaruref{ex:double-schwa} is notable for being one of the few situations utilizing \orthotext{ee}. It is preserved in some words to distinguish otherwise opaque stress or as an etymological consequence such as in \baaruref{ex:double-schwa-erg}.

\section{Derivational Morphology}
\begin{description}
  \item[-la] derives from an adjective a noun meaning \transtext{composed from X}
  \item[-ku] constructs a verb with the meaning \nativetext{ewi-siha hasi X ta} \transtext{to happen as if by X, to appear as X}, where X is a noun
\end{description}

\subsection{-la}

The primary and perhaps most straightforward use of \nativetext{-la} is that of material composition:

\baabbrev{adjz}{adjectivizer}
\begin{examples}
  \baarucols{2}
  \ex
    \preamble lelu sla
    \pronunciation lə.ˈlu ˈsu.xə.la
    \gloss
      lelu & spoon \\
      suxe & \MC2 processed\_wood-ADJZ \\
      -la & \\
    \tr wooden spoon
  \ex
    \preamble kele wasila
    \pronunciation kə.ˈle ˈwa.si.la
    \gloss
      kele & bowl \\
      wasi & river\_stone-ADJZ \MC2 \\
      -la & \\
    \tr stone bowl
\end{examples}

\nativetext{-la} should not be confused with the uses of the English preposition \transtext{of}.

\begin{example}
  \preamble ti'ika kitala
    \pronunciation ˈti.ʔi.ka ˈki.ta.la
    \gloss
      ti'ika & branch \\
      kita & raw\_wood-ADJZ \MC2 \\
      -la & \\
    \tr wooden branch
    \intended tree branch
\end{example}

This particular phrase calls for the associative particle \nativetext{'e}, which achieves the desired semantic relationship. This will be further detailed in \ref{ch:morpho-adj}; the correct formation is illustrated here:

\baabbrev{assoc}{associative}
\begin{example}
  \preamble ti'ika'e kita
    \pronunciation ˈti.ʔi.ka.ʔə ˈki.ta
    \gloss
      ti'ika & branch-ASSOC \MC2 \\
      -'e & \\
      kita & tree \\
    \tr tree branch
\end{example}

Apart from this, \nativetext{-la} can also designate the origin of a person or idea. Notably, it refers to generic origination rather than circumstantial provenance.

This distinction manifests in \langname 's strategies for expressing residency.

\aside{While it is expected that a \nativetext{la}-toponym refer to one's birthplace, this is not univerally true in colloquial usage.}
\aside{In the case of young children who grow up in a different place than they were born, \nativetext{la}-toponyms may shift to the more formative, newer location.}

%? Uakci what should I call 'la-forms' instead? Currently reads a bit weird
%?? TODO – ponder on this more ig

\baabbrev[\FIRST]{1}{first person}
\baabbrev{name}{name}
\begin{examples}
  \baarucols{2}
  \ex
    \preamble me ɸene wu ehaixa
    \pronunciation ˈme ɸə.ˈne ˈwu ə.ˈhai.xa
    \gloss
      me & 1sg \\
      ɸene & live \\
      wu & on\_top \\
      Ehaixa & NAME \\
    \tr I live in Ehaixa.
  \ex
    \preamble me ehaixala
    \pronunciation ˈme ə.ˈhai.xa.la
    \gloss
      me & 1sg \\
      Ehaixa & NAME-ADJZ \MC2 \\
      -la & \\
    \tr I am from Ehaixa.
\end{examples}
