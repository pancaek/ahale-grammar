\chapter{Verbal}

Morphologically, verbs are quite simple. Tense is split into past and non-past forms, while aspect is split simply along perfectivity. This results in a simple set of four possible inflections, three of which are overtly marked.

\begin{table}[ht]
  \centering
  \begin{tabular}{*{3}{c}}
    \toprule
                 & Nonpast & Past  \\\midrule
    Imperfective & ∅-      & i-    \\
    Perfective   & V-      & m(u)- \\
    \bottomrule
  \end{tabular}
  \caption{Verb Inflection}
  \label{table:verb-inflection}
\end{table}

The \NPST.\PFV\ form can be described as \morphtext{V,}, where V is a copy vowel dependent on the adjacent syllable. For this reason, i-stem verbs will display a minimal pair between initial- and second-syllable stress in the \NPST.\PFV\ and \PST.\IPFV\ forms.

When a verb stem begins in a vowel, epenthetic \phontext{ʔ} will be inserted for both the \NPST.\PFV\ and \PST.\IPFV\ forms.
