\chapter{Verbal}


In \langname , verb inflection is extremely minimal. Most verbal infornation is conveyed through peripherastic constructions, the most common being multiple-verb constructions.

\begin{table}[ht]
  \centering
  \begin{tabular}{*{3}{c}}
    \toprule
                 & Nonpast & Past  \\\midrule
    Imperfective & ∅-      & i-    \\
    Perfective   & V(ʔ)-   & m(u)- \\
    \bottomrule
  \end{tabular}
  \caption{Verb Inflection}
  \label{table:verb}
\end{table}

The \textsc{npst.pfv} form is special. The vowel inserted is dependent on the nucleus of the syllable it attaches to. V `echoes' from adjacent syllables. Thus, a form such as \langword{ama,} conjugated in the \textsc{npst.pfv} form, becomes \langword{aʔama.}

Similarly, a form with an initial consonant such as \langword{litu} will have its nucleus echoed to produce \langword{ilitu} as the \textsc{npst.pfv} form. But it also brings up the question, how does this differ from the \textsc{pst.ifv} form? This echo vowel behaves slightly differently than a typical \langname\ prefix. It is unable to be stressed, which in turn means that \langword{litu} will be realized as [ˈi.li.tu] in the \textsc{pst.ipfv} form, but [iˈli.tu] in the \textsc{npst.pfv} form. These minimal pairs only occur with i-stem verbs.

\section{Alignment}

Verbs follow direct-inverse alignment, which utilizes a person hierarchy to determine the appropriate verb marking. The direct construction is used when the agent of the transitive clause outranks the patient in the person hierarchy, and the inverse is used when the patient outranks the agent. It should be made clear that this type of alignment coexists with the ergative-absolutive alignment of nouns.

\subsection{Person Hierarchy}\label{sec:person_hierarchy}
One of the core mechanisms of a direct-inverse system is its person hierarchy. As mentioned above, this determines if the verb will be used in the direct or inverse form, based on the relative positions of verbal arguments in the hierarchy.

The person hierarchy in \langname\ is: 2\textsuperscript{nd} person > 1\textsuperscript{st} person > 3\textsuperscript{rd} person proximate (\textsc{prox}) > third person obviative (\textsc{obv}).

The proximate/obviative distinction is particularly notable. It is used to disambiguate situations where both arguments of the verb are 3\textsuperscript{rd} person. It is not uncommon for the proximate/obviate marking to be dropped, due to functional overlap with the noun cases that \textit{do} exist.

\pex<alignment-dir>
\begin{paracol}{2}
\a<grammatical>
\begingl
\glpreamble tutu kula me
\pronounced{ˈtu.tu ˈku.la ˈme}\endpreamble
tu\textasciitilde tu[\textsc{erg\textasciitilde 2sg}]
kula-∅[hurt-\textsc{dir}]
∅-me[\textsc{abs-1sg}]
\glft `You are hurting me.'
\endgl
\switchcolumn
\a<ungrammatical>
\begingl
\glpreamble \ljudge{*} du kula meme
\endpreamble
∅-tu[\textsc{abs-2sg}]
kula-∅[hurt-\textsc{dir}]
me\textasciitilde me[\textsc{erg\textasciitilde 1sg}]
\endgl
\end{paracol}
\xe


\pex<alignment-inv>
\begin{paracol}{2}
\a<grammatical>
\begingl
\glpreamble meme kulasi tu
\pronounced{ˈme.mə ˈku.la.si ˈtu}\endpreamble
me-me[\textsc{erg\textasciitilde 1sg}]
kula-si[hurt-\textsc{inv}]
∅-tu[\textsc{abs-2sg}]
\glft `I am hurting you.'
\endgl
\switchcolumn
\a<ungrammatical>
\begingl
\glpreamble \ljudge{*} me kulasi tutu
\endpreamble
me[\textsc{abs-1sg}]
kula-si[hurt-\textsc{inv}]
tu-tu[\textsc{erg\textasciitilde 2sg}]
\endgl
\end{paracol}
\xe

The latter parts of these examples are ungrammatical, because the case marking on the nouns implies roles which are opposed by the marking on the verbs.

The direct-inverse marking is not entirely redundant, as it may seem from the above examples. The overlap of these systems allows some pronouns to be dropped, with no loss in clarity. For example, Example \getfullref{alignment-dir.grammatical} may also be conveyed as follows:

\ex
\begingl
\glpreamble kula me
\pronounced{ˈku.la ˈme}\endpreamble
kula-∅[hurt-\textsc{dir}]
∅-me[\textsc{abs-1sg}]
\glft `You are hurting me.'
\endgl
\xe

There is no explicitly stated agent in this example, but it is not necessary in this instance. The verb \langword{kula} is in its direct form, meaning its agent outranks its patient in the person hierarchy. Only one thing outranks a 1\textsuperscript{st} person patient, that being a 2\textsuperscript{nd} person agent.

In a similar fashion, the object of an obviously transitive verb may be dropped as well:

\ex
\begingl
\glpreamble kula!
\pronounced{ˈku.la}\endpreamble
kula-∅[hurt-\textsc{dir}]
\glft `You are hurting me.'\\`It hurts!'
\endgl
\xe

Because of the origin of this construction, this may not be used in the same general way `It hurts!' can be in English, though \langword{kula!} is still translated as such based on context.

\section{Tense and Aspect}
\subsection{Constructing the Future}

\langname\ makes no morphological distinction between present and future tense. In everyday discourse, its speakers avoid explicitly referring to the future, as it is seen as overly speculative for most purposes. This is especially noticable when the reoccurence of a habitual action is questioned. To exemplify this with an English example, a speaker of \langname\ will prefer ``The sun rises?'' to a similar ``Will the sun rise?''. Self evident truths such as these do not require additional tense marking, as the interrogative fills the same role.

\subsubsection{Habitual Imperfectives}

The above question and and answer would be rendered as follows in \langname :

\ex<ex:future_interrrogative>
\begingl
\glpreamble ti masa
\pronounced{ˈti ˈma.sa}\endpreamble
∅-ti[\textsc{npst.ipfv-}rise\textsc{[q]}]
masa[sun]
\glft `The sun rises?'\\`Will the sun rise?'
\endgl
\xe

Note that even though this event may be typically rendered with the perfective aspect, the imperfective is used instead. This is because the rising of the sun is known to be habitual, and so the imperfective is used to show the knowledge of this. This is not typical, but is commonly done when referring to things that happen out of human control, thimgs which are ``just the way the world works''. Phenomena which follow this priciple include:

\begin{itemize}
  \item The passage of seasons
  \item Cycles of the sun and moon
  \item Other cyclic natural processes
  \item Time, in the context of inevitability and continual change
\end{itemize}

Now that the question has been constructed, we must construct an answer. The SVO order, typical of declarative sentences, is returned to. In a simple example such as this one, the rest of the sentence remains unaltered. And thus, an \langname\ speaker will simply reply:

\ex<ex:implied_future>
\begingl
\glpreamble masa ti
\pronounced{ˈma.sa ˈti}\endpreamble
masa[sun] ∅-ti[\textsc{npst.ipfv-}rise]
\glft `The sun rises.'
\endgl
\xe

These imperfectives continue to be read as specifically habitual, even in declarative sentences. To convey a progressive reading, a duration must be specified.
It should also be noted that units of time have a implicit quantity of one, and so no explicit mention of quantity of hours is necessary in this example.

\ex
\begingl
\glpreamble masa neʔe ti
\pronounced{ˈma.sa ˈne.ʔə ˈti}\endpreamble
masa[sun]
neʔe[hour]
∅-ti[\textsc{npst.ipfv-}rise]
\glft `The sun has been rising for one hour.'
\endgl
\xe

\ex
\begingl
\glpreamble masa pa ti
\pronounced{ˈma.sa ˈpa ˈti}\endpreamble
masa[sun]
pa[now]
∅-ti[\textsc{npst.ipfv-}rise]
\glft `The sun is rising.'
\endgl
\xe

\subsubsection{Explicit Future}
On the rare occasion that the future must be explicitly marked, a different construction can be used. \langname\ has as expression which loosely translates to ``It was, it is [therefore it must always be]''. This implied segment allows similar constructions to be used in a grammatical context, as well as in discourse.

This construction can be used with almost any verb, but it is avoided due to its formality, and doubly so because an archaic copula, \langword{wa.}

\ex
\begingl
\glpreamble iwa, alete wa
\pronounced{ˈi.wa, ˈa.lə.te ˈwa}\endpreamble
i-wa,[\textsc{pst.ipfv-cop}]
alete[thus]
∅-wa[\textsc{npst.ipfv-cop}]
\glft `It was, it is, [therefore it must always be].'
\endgl
\xe

The copula used in this phrase is another of these verbs, as it refers more to the ongoing passage of time rather than the event itself (if it were, \langword{wa} would not be used in this fashion).


Here, Example \getfullref{ex:implied_future} is rendered explicitly in the future, using \textit{wa-repetition.}
\ex<ex:explicit_future>
\begingl
\glpreamble masa iti, alete ti
\pronounced{ˈma.sa ˈi.ti | ˈa.lə.te ˈti}\endpreamble
masa[sun]
i-ti[\textsc{pst.ipfv-}rise,]
alete[thus]
∅-ti[\textsc{npst.ipfv-}rise]
\glft `The sun will rise.'
\endgl
\xe

The most notable exception is when discussing emotion, in which case this construction cannot be used at all.

\ex<ex:emotion_wrong>
\begingl
\glpreamble \ljudge{*} inale, alete nale
\pronounced{ˈi.na.le | ˈa.lə.te ˈna.lə}
\endpreamble
i-nale,[\textsc{pst.ipfv-}be\_sad]
alete[thus]
∅-nale[\textsc{npst.ipfv-}be\_sad]
\glft `I will be sad.'
\endgl
\xe

Due to the sense of permanence the construction creates, it is ungrammatical to use here. It may even sound to some to be a malformed causative, but this is neither grammatically nor semantically correct. Emotions in \langname\ are treated as only things felt. Although a person can \textit{do things} to cause an emotional response, they cannot cause them directly. This idea is reflected in the constructions used in reference to emotion.

The proper way to convey Example \getfullref{ex:emotion_wrong} is shown here:

The usual strategy of \textit{wa-repetition} for dynamic verbs is not applicable for stative verbs, and as such a different strategy must be used. Stative verbs can only be explicitly placed into the future tense through the use of an auxilliary verb, \langword{sihu} placed before the main verb.

\ex
\begingl
\glpreamble sihu nale
\pronounced{ˈsi.hu ˈna.lə}\endpreamble
Ø-sihu[\textsc{npst.ipfv-}happen]
Ø-nale[\textsc{npst.ipfv-}be\_sad]
\glft `I will be sad.'
\endgl
\xe

In discourse, this construction (from now on referred to as \textit{wa-repetition}) has a similar, but distinct meaning to its grammatical counterpart. Any one of ``So be it.'', ``It is what it is.'', or sometimes even ``Leave it be.'' may be apt translations. In these situations, the phrase may be shortened to `iwalete', although this is seen as incredibly informal and potentially rude.

\section{Mood}
\subsection{Expressing Interrogatives}

\langname 's interrogative is expressed through a change in word order. The verb is fronted, and conjugated as would be expected.

\ex
\begingl
\glpreamble muti xanu
\pronounced{ˈmu.ti ˈxa.nu}\endpreamble
mu-ti[\textsc{pst.pfv-}rise\textsc{[q]}]
xanu[bear]
\glft `Did the bear wake up?'
\endgl
\xe

The type of question being asked is often left up to context, and is usually translated using English `do' or `how' when there is insufficent context to make a finer distinction. Ambiguous questions, as seen here, also form polar questions. Deliberatly underspecifying these is done as well.

\ex
\begingl
\glpreamble mai muti xanu
\pronounced{ˈmai ˈmu.ti ˈxa.nu}\endpreamble
mai[time]
mu-ti[\textsc{pst.pfv-}rise\textsc{[q]}]
xanu[bear]
\glft `When did the bear wake up?'
\endgl
\xe

Several things should be noted from this example. \langname\ does not have dedicated wh-forms, instead using nouns for this purpose, which behave somewhat like an adverb, specifying what specifically is being questioned. he words used in this clarification can function as typical nouns as well. The use of these nouns is not mandatory, however, as the nature of the question can sometimes be inferred through context. When nouns are used as clarification of interrogatives, they are placed directly before the verb, and left unmarked for case. This is significant in that \langword{muti} is being used intransitively, rather than in a transitive sense with \langword{mai} as an argument of the verb.

\ex
\begingl
\glpreamble mamai uhumi xaku
\pronounced{ˈma.mai uˈɣu.mi ˈxa.ku}\endpreamble
ma\textasciitilde mai[\textsc{erg\textasciitilde}time]
u-humi[\textsc{npst.pfv-}heal]
∅-xaku[\textsc{abs-}pain]
\glft `Time heals all wounds.'
\endgl
\xe
In this example, \langword{uhumi} `to heal', has \langword{mai} as an argument. It is being used transitively, as evidenced by the direct object \langword{xaxaku} and the presence of ergative marking. Though \langword{mai} precedes the verb, there is no interrogative reading of this example.

Many words can be used to clarify the nature of an interrogative, far more than the interrogative pronouns of English. Here are just a few examples of how this can be used:

\begin{paracol}{2}
\ex
\begingl
\glpreamble laʔia xanu
\pronounced{ˈlaʔia ˈxa.nu}\endpreamble
laʔia[color]
xanu[bear]
\glft `What color is the bear?'
\endgl
\xe
\switchcolumn

\ex
\begingl
\glpreamble kane xanu
\pronounced{ˈka.nə ˈxa.nu}\endpreamble
kane[strength]
xanu[bear]
\glft `How strong is the bear?'
\endgl
\xe
\end{paracol}