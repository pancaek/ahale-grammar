\section{Attributives}

\Cref{ex:attributive-cameo} briefly mentions attributives with regards to \highlight{\rom{ali'a nu}}, but this section will provide a much more comprehensive explanation.
\subsection{In summary:}
The associative particle \rom{'e} is suffixed onto the head, while the nominals following function attributively. This results in the structure \textsc{head}-'e \textsc{attr}. The application of this construction is far from universal; the following highlights key indicators of (in)applicability:
\begin{itemize}
  \item Attributes may not be names (personal or otherwise)
  \item Attributes must be able to stand on their own
  \item Attributes may be denominations of general dimensions of characteristics
  \item Attributes may only be specific values of a dimension (characteristics, objects) in comparative use
  \item Attributes must be culturally established concepts; the \rom{'e} associative relies on the speaker being able to make assumptions about a familiar concept
\end{itemize}

Further, the exact function of \rom{'e} can be broadly split into three categories: \detail{general, serial,} and \detail{nesting.}

\subsection{General attribution}
In the most basic sense, \rom{'e} serves to highlight the prescence of \textsc{attr}, doing so in a ``positive'' direction. This is determined by implications of \detail{specificity} and \detail{familiarity}.

For example, trees are intrinsically assumed to be tall, thus highlighting height describes an idealized, particularly tall tree.

\begin{example}
  \preamble kita'e pala
  \gloss
    kita & tree \\
    -'e & ASSOC \\
    pala & height \\
  \tr tall tree
\end{example}

Recall that this relies on pre-established cultural notions; attempting attribution otherwise results in an infelicitous phrase. Bears, for example, hold little significance as pertaining to height, and therefore little can be determined by highlighting this.

\begin{example} %! \baarujudge{\#} is broken; this is a workaround
  \preamble:\# xanu'e pala
  \gloss
  xanu & bear \\
  -'e & ASSOC \\
  pala & height \\
  \tr tall bear
\end{example}


The association done through \rom{'e} is not always literal. Particularly, descriptions of natural phenomena may utilize a more idiomatic \rom{'e}:

\begin{example}
  \preamble nauwa'e sika
  \gloss
    nauwa & flowing\_water \\
    -'e & ASSOC \\
    sika & mischief \\
  \tr turbulent water
  \lit mischievious water
\end{example}

\subsection{Serial attribution}

\textsc{assoc} may be used serially to characterize a noun as a particular subtype of itself. This use is only applied to series' of different variants of one head. For single instances of \textsc{attr} in this situation, \textsc{head} and thus \textsc{assoc} are unnecessary and omitted.

The resulting structure can be generalized as \textsc{head\textsubscript{1}-assoc attr\\head\textsubscript{1}-assoc attr\textsubscript{2}}

\begin{example}
  \preamble kita'e ??? kita'e ??? pasini
  \gloss
    kita & tree \\
    -'e & ASSOC \\
    ??? & pine\_tree \\
    kita & tree \\
    -'e & ASSOC \\
    ??? & oak\_tree \\
    pasini & grow \\
  \tr Pine and oak trees are growing.
\end{example}

Whereas, when a single type is referenced in isolation, the bare noun is used:

\begin{example}
  \preamble a'au lau ???
  \gloss
    a'^ & ERG \\
    au & 1 \\
    lau & neaten \\
    ??? & pine\_tree \\
  \tr I am trimming the pine tree.
\end{example}

\subsection{Nested attribution}

This structure can be further extended in order to form the nested use, which further exemplifies this noteworthiness. This requires that the same attribute be used as both the inner and outer description of the head. This can be used either in description of a sequence of similar things, or in isolation simply for emphatic purposes.

This can be generalized to a pattern: [\textsc{head-}\textsc{assoc} \textsc{attr}]-\textsc{assoc} \textsc{attr}

To return to our tree example, it is not uncommon for a construction as follows to be made:

\begin{example}
  \preamble kita'e pala'e pala
  \gloss
    [ \\
    kita & tree \\
    -'e & ASSOC \\
    pala ] & height \\
    -'e & ASSOC \\
    pala & height \\
  \tr taller-than-tall tree
\end{example}
