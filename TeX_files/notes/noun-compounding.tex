\section{Compound nominals}
\subsection{with conjunctions \rom{ke} and \rom{a}}

Two of the most common conjunctions are \rom{ke} and \rom{a}, which link nominals. \rom{Ke} is a comitative conjunction, usually translated as \trans{and}.

\begin{example}
  \romanization a'u ke mele
  \gloss
    a'u & NDST:1 \\
    ke & COM \\
    mele & older\_parent \\
  \tr My parent and I
\end{example}

Coordinated phrases must be explicitly separated, \rom{ke} cannot be implied as sometimes occurs in English here. It can also be used for listing nouns, without the presence of a verb.

\begin{example}
  \romanization a'au ke memele keke upe
  \gloss
    a & ERG^ \\
    au & NDST:1 \\
    ke & COM \\
    me & ERG^ \\
    mele & older\_parent \\
    keke & eat \\
    upe & yam
  \tr My mom and I eat yams.
\end{example}

With \rom{ke}, there is an implication that the meal is shared. Alternatively, it can be used to emphasize that \detail{all} participants eating is requisite to another event.

\rom{A} is a distributive, but it serves a different function from the distributive affix \rom{-me-}. Instead of describing the noun directly, \rom{a} primarily describes how agentive nouns relate to an associated verb phrase. As a consequence of this, \rom{a} cannot be used without a verb.

\begin{example}
  \romanization a'au a memele keke upe
  \gloss
    a' & ERG^ \\
    au & NDST:1 \\
    a & DIST \\
    me & ERG^ \\
    mele & older\_parent \\
    keke & eat \\
    upe & yam
  \tr My mom and I eat yams.
\end{example}

This example's use of \rom{a} shows that both people are doing the task, but not necessarily at the same time. When used with a sequence of events, the relative timing of each participant is unspecified. Sometimes, it is useful to treat this as a recent past for translation purposes, because English simple present often resembles a habitual.
