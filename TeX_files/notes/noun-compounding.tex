\section{Compound nominals}
Generally speaking, multiple NPs will not follow each other in default position, and will be marked with \native{立/呂} when focus moves NPs adjacent to each other.
Compound nominals can be formed through intentional adjacency of nouns (forming \highlight{adjacency compounds}), which can be seen in \cref{ex:ndst-pred-cop}. In this situation, the first noun is treated as the head, while the following nouns progressively narrow the meaning.
In the case of \cref{ex:ndst-pred-cop} this is roughly analogous to implicit `of' between the component nouns (body [of] bird), but the use is wider in \langname\ than it is in English.

Namely, to narrow or shift meanings by chaining near synonyms or semantically related concepts:
\begin{example}
  \romanization wasi nauwa
  \gloss
    wasi & river\_stone \\
    nauwa & flowing\_water \\
  \tr stones (from the riverbed)
\end{example}

\begin{example}\label{ex:adjacency-chain}
  \romanization wasi nauwa ileu'ei
  \gloss
    wasi & river\_stone \\
    nauwa & flowing\_water \\
    ileu'ei & origin \\
  \tr stones (from the mouth of the river)
  \alt waystones (which mark the river)
\end{example}

\Cref{ex:adjacency-chain} shows an interesting feature of adjacency compounds, this being that the relationship between dependent nouns is not necessarily linear. Aside from previous context, the precise meaning of these compounds can also be distinguished through prosody.

Generally speaking, the pitch of simple noun phrases will decrease steadily. This is because noun phrases are head-initial, with the head receiving the highest pitch.

\begin{contour}
\preamble [wasi [nauwa] [ileu'ei]]
\text <3:w>asi nauwa ileu'e<1:i>
\tr stones (from the mouth of the river)
\end{contour}

In a situation where a noun phrase is itself composed of several nested phrases, the first head still receives the highest pitch. However, nested heads also have slightly raised pitch--- causing slight ``steps'' in the overall contour.


\begin{contour}
\preamble [wasi [nauwa [ileu'ei]]]
\text <3:w>as<1.75:i> <2.75:n>auwa ileu'e<1:i>
\tr waystones (which mark the river)
\end{contour}

When modifying these compounds with adjectives, role marking particles must be used to distinguish the noun from following nominals serving as adjectives:

\begin{example}
  \romanization wasi nauwa (ta'e) ti'u
  \gloss
    wasi & river\_stone \\
    nauwa & flowing\_water \\
    ta & ta \\
    -'e & ASSOC \\
    ti'u & unevenness\sidenotemark \\
  \tr pointed stones (from the riverbed)
\end{example}

\sidenotetext{\rom{Ti'u} specifically refers to unevenness which favors a particular side over another. Teardrop shapes are perhaps a good example to distinguish from unevenness built upon inconsistency. Teardrop shapes can be described with \rom{ti'u} because of the contrast of the pointed end to the smooth bulb. Long and flat rocks may also be described this way, particularly if they are oblong. Unevenness as related to texture or consistency is translated instead through \rom{hitexu}, an ideophone which mimics various types of scraping noises.}
