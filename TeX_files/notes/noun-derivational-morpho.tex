\section{Derivational Morphology}
\begin{description}
  \item[-la] derives from an adjective a noun meaning \transtext{composed from X}
  \item[-ku] constructs a verb with the meaning \nativetext{ewi-siha hasi X ta} \transtext{to happen as if by X, to appear as X}, where X is a noun
\end{description}

\subsection{-la}

The primary and perhaps most straightforward use of \nativetext{-la} is that of material composition:

\begin{examples}
  \baarucols{2}
  \ex
    \preamble lelu suxela
    \pronunciation lə.ˈlu ˈsu.xə.la
    \gloss
      lelu & spoon \\
      suxe & processed\_wood \\
      -la & ADJZ \\
    \tr wooden spoon
  \ex
    \preamble kele wasila
    \pronunciation kə.ˈle ˈwa.si.la
    \gloss
      kele & bowl \\
      wasi & river\_stone \\
      -la & ADJZ \\
    \tr stone bowl
\end{examples}

\nativetext{-la} should not be confused with the uses of the English preposition \transtext{of}.

\begin{example}
  \preamble ti'ika kitala
    \pronunciation ˈti.ʔi.ka ˈki.ta.la
    \gloss
      ti'ika & branch \\
      kita & raw\_wood \\
      -la & ADJZ \\
    \tr wooden branch
    \intended tree branch
\end{example}

This particular phrase calls for the associative particle \nativetext{'e}, which a\-chieves the desired semantic relationship. This will be further detailed in \ref{ch:morpho-adj}; the correct formation is illustrated here:

\begin{example}
  \preamble ti'ika'e kita
    \pronunciation ˈti.ʔi.ka.ʔə ˈki.ta
    \gloss
      ti'ika & branch \\
      -'e & ASSOC \\
      kita & tree \\
    \tr tree branch
\end{example}

Apart from this, \nativetext{-la} can also designate the origin of a person or idea. Notably, it refers to origination which is generic rather than circumstantial.

This distinction manifests in \langname 's strategies for expressing residency.

\aside{While it is expected that a \nativetext{la}-toponym refer to one's birthplace, this is not univerally true in colloquial usage.}
\aside{In the case of young children who grow up in a different place than they were born, \nativetext{la}-toponyms may shift to the more formative, newer location.}

%? Uakci what should I call 'la-forms' instead? Currently reads a bit weird
%?? TODO – ponder on this more ig

\begin{examples}
  \baarucols{2}
  \ex
    \preamble me ɸene wu ehaixa
    \pronunciation ˈme ɸə.ˈne ˈwu ə.ˈhai.xa
    \gloss
      me & 1sg \\
      ɸene & live \\
      wu & on\_top \\
      Ehaixa & NAME \\
    \tr I live in Ehaixa.
  \ex
    \preamble me ehaixala
    \pronunciation ˈme ə.ˈhai.xa.la
    \gloss
      me & 1sg \\
      Ehaixa & NAME \\
      -la & ADJZ \\
    \tr I am from Ehaixa.
\end{examples}

As mentioned above, \nativetext{-la} can also be used to describe the origin of an idea or concept. This can be used formally, in the sense of invention, or simply to describe where a thought comes from.

This is found primarily within conversational, reportative discourse, rather than in formal accounts or official notices.

\aside{Auku is most often translated as \transtext{hear}, though it is also used as a general sensory verb more akin to \transtext{feel}. However, \nativetext{auku} is not used in the context of emotion.}
\begin{subexamples}
  \ex
    \preamble pane auku tu (kini)
    \pronunciation ˈpa.nə ˈau.ku ˈtu (ˈki.ni)
    \gloss
      pane & person \\
      auku & hear \\
      tu & 2SG \\
      ( kini ) & this \\
    \tr Where did you hear that?
    \alt Who told you that?
  \ex
    \preamble nalexu tala
    \pronunciation ˈna.lə.xu ˈta.la
    \gloss
      Nalexu & NAME \\
      ta & PROX \\
      -la & ADJZ \\
    \tr Nalexu did!
\end{subexamples}

When \nativetext{-la} would be attached to names, particularly those of people, it is instead attached to the corresponding focus marker.

\aside{This stems from a widespread avoidance of inflecting proper nouns. Similar avoidance strategies can be found across a variety of constructions.}

\notabletext{Formal use of \nativetext{-la}} is typically reserved for description of relationships which are for the most part, indisputable. It can express a particular type of ownership most easily correlated with geographical location, but it is also used in a more political sense in reference to whole nations.

\begin{example}
  \preamble nauwa ateenala
  \pronunciation ˈnau.wa a.ˈte.na.la
  \gloss
    nauwa & flowing\_water \\
    Ateena & NAME \\
    -la & ADJZ \\
  \tr Ateenalen water
\end{example}

\subsection{-ku}

\aside{\nativetext{-ku} comes from a colloquial shortening of the verb \nativetext{ku'ime} \transtext{to attempt}.}

\nativetext{-ku} in isolation derives an intransitive verb, most often used to describe the manner by which domething occurred.

It's meaning has shifted since the construction originally diverged, resulting in the additional hedging found in the present definition. Rather than the outcome being unspecified, as with \nativetext{ku'ime}, \nativetext{-ku} generally imparts that the attempt did not go as expected. If successful, it will convey dishonesty or trickery in the resulting achievement.

\aside{\nativetext{Auku} is generally used as a sensory verb, but consequently it can be zero-derived to a noun roughly meaning \transtext{knowledge, experience, skill}.}

\begin{example}
  \preamble anaku, eusaku, alete auku'u
  \gloss
    ana & eye-VBZ \MC2 \\
    -ku & \\
    eusa & ear-VBZ \MC2 \\
    -ku & \\
    alete & thus \\
    auku & knowledge-VBZ \MC2 \\
    -'u & \\
    \tr I looked, I listened, but I couldn't find anything.
  \end{example}

  %* Remember to note recent past use with alete + present tense

  In this example, \nativetext{-ku} simply conveys unsuccessful action. Notably, no overt negation is present. This is a feature of this derivation, especially in conjunction with \nativetext{alete}.

\begin{subexamples}
  \baarucols{2}
  \ex
    \preamble wa meme suxi!
    \pronunciation ˈwa mə.ˈme ˈsu.xi
    \gloss
      wa & AFF \\
      me\allo me & ERG\allo 1SG \\
      suxi & solve \\
    \tr I figured it out!
  \ex
    \preamble ta auku'u
    \pronunciation ˈta ˈau.ku.ʔu
    \gloss
      ta & PROX \\
      auku & knowledge-VBZ \MC2 \\
      -'u & \\
    \tr Of \textit{course} you did...
\end{subexamples}

This exchange highlights the negative connotations of \nativetext{-ku}, particularly in its use in conveying trickery. The second speaker believes the first is not smart enough for such critical thinking. While it \textit{is} a fairly generic insult, it is partially contextually sensitive.


This derivation is most productively used within compound verbs.

\aside{\nativetext{Lasaka'ane} is derived from compounding of \nativetext{lasaka} \transtext{choice} + \nativetext{pane} \transtext{person}}

\begin{examples}
  \baarucols{2}
  \ex
    \preamble lasaka'aneku suxi
    \gloss
      lasaka'ane & council-VBZ \MC2 \\
      -ku & \\
      suxi & solve \\
    \tr to come to a mutual understanding
  \ex
    \preamble lasaka'aneku
    \gloss
      lasaka'ane & council-VBZ \MC2 \\
      -ku & \\
    \tr to debate
\end{examples}