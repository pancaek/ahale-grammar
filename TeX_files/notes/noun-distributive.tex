\section{Distributives}
Rather than describing nouns by their number, \langname\ describes nouns as either \highlight{distributive} or \highlight{non-distributive}.
Uses of the distributive form can be roughly divided into the following categories:

\begin{itemize}
  \item To form collective plurals
  \item To form plurals
  \item For prototypical distributive use
  \item To make generalizations
\end{itemize}

\subsection{Collective vs.\ collective plural}
Inherently dual nouns and collective nouns \nativeparen{物有二} may be modified by \DST\ to refer to multiple \detail{complete} sets.

% \begin{examples}
%   \baarucols{2}
%   \ex
%     \script 手
%     \romanization heu
%     \gloss
%       ∅- & NDST \\
%       heu & hand \\
%     \tr (Both) hands
%     \not (Two) hands
%   \ex
%     \script 各手
%     \romanization meheu
%     \gloss
%       <me>heu & <DST>hand \\
%     \tr Sets of hands
% \end{examples}

% \subsection{As a prototypical distributive}
% \begin{example}\label{ex:dst-pred-cop}
%   \script 各体鳥立垢
%   \romanization menula sipu ta feisa
%   \gloss
%     <me>nula & <DST>body \\
%     sipu & bird \\
%     ta & ta \\
%     feisa & filth \\
%   \tr These bird carcasses (of many types) are so gross.
% \end{example}

% This sentence may appear to be a generalization of the qualities of bird carcasses.
% However, generalized readings are more common with verbs inherently descriptive of opinion, rather than the predicational copular phrase \labelcref{ex:dst-pred-cop} presents.
% \Cref{ex:dst-pred-cop} also acknowledges that under different circumstances, the bird carcasses \detail{may be perceived differently}.

% \begin{example}\label{ex:ndst-pred-cop}
%   \script 体鳥立垢
%   \romanization nula sipu ta feisa
%   \gloss
%     nula & body \\
%     sipu & bird \\
%     ta & ta \\
%     feisa & filth \\
%   \tr These bird carcasses are so gross.
% \end{example}

% Being the non-distributive form of \labelcref{ex:dst-pred-cop}, \cref{ex:ndst-pred-cop} refers to a single type of carcass. Number is unspecified, instead referring to \detail{specific referents} established with prior context.

% \subsection{Making generalizations/expressing opinions}
% As mentioned previously, the use of \morphtext{me} to express opinion is mostly confined to its use alongside verbs describing emotion or state of mind.

% \begin{examples}
%   \ex
%     \romanization a'au a'aeu kita'e auka
%     \gloss
%       a'\allo & ERG \\
%       au & 1 \\
%       a'aeu & enjoy\_presense \\
%       ∅- & NDST \\
%       kita & tree \\
%       -'e & ASSOC \\
%       auka & oak \\
%     \tr I like those oak trees.
%   \ex\label{ex:generalization-trees}
%     \romanization a'au a'aeu mekita'e auka
%     \gloss
%       a'\allo & ERG \\
%       au & 1 \\
%       a'aeu & enjoy\_presense \\
%       <me>kita & <DST>tree \\
%       -'e & ASSOC \\
%       auka & oak \\
%     \tr I like oak trees.
% \end{examples}

% Generalized statements are often made by omitting personal pronouns, and optionally emphasizing the topic of the opinion.

% In situations such as \labelcref{ex:generalization-trees} where the agent is making the generalization, the process of emphasizing the opinion is quite simple, because the patient can simply be fronted:

% \begin{example}
%   \romanization mekita'e auka a'aeu
%   \gloss
%     <me>kita & <DST>tree \\
%     -'e & ASSOC \\
%     auka & oak \\
%     a'aeu & enjoy\_presense \\
%   \tr Oak trees are pleasant.
% \end{example}
% \filbreak

% This changes slightly when the agent is the opinion's topic. If further fronting cannot occur, \native{立} is used to emphatically mark agency:

% \begin{example}
%   \romanization ta simesikama aseisi
%   \gloss
%     ta & ta \\
%     si\allo & ERG\\
%     <me>sikama & <DST>buzzing\_insect \\
%     asei & scare \\
%     -si & INV \\
%   \tr Insects are scary.
%   \lit Insects scare (me).
% \end{example}

% An exception to the verbal restrictions surrounding generalization occurs when expressing opinions about the \detail{attributes}\footnote{The attribute must be \native{質自} \rom{ali'a nu} \trans{an inherent quality} of the object being described, and must appear alongside the associative particle, \rom{'e}.} of the an object.%
% \marginpar{Qualities which can be described this way are idealized or culturally significant.
% Opinions on these qualities are seen as more important, and the choice to describe them this way gives the opinion more weight than otherwise imparted by a more peripherastic construction.}

% \begin{example}
%   \script 各木亜脊立煩
%   \romanization mekita'e pala ta sika
%   \gloss
%     <me>kita & <DST>tree \\
%     -'e & ASSOC \\
%     pala & height \\
%     ta & ta \\
%     sika & worry \\
%   \tr Tall trees are foreboding.
% \end{example}
% \subsection{Singular vs.\ plural}
% The plural reading of \morphtext{me} occurs with personal pronouns, foreign loans that are not \native{物有二}, or loans which have not been sufficiently nativized, as with \labelcref{ex:nativized-loans}.
% Established loans, such as \rom{i'isi} \trans{foreigner} follow the prototypical distibutive reading.

% \begin{columns}[cols.markup=\mutations]\label{ex:nativized-loans}
%   \cols leunu & meleunu & cactus/cactuses & (Lérvima \rom{lonón}) \\
%   \cols neke & meneke & goat/goats & (Epe \rom{ŋehke})
% \end{columns}
