\section{Inflection and morphophonology}
\subsection{Case}
\langname\ has only two cases: the ergative and the absolutive.
The ergative case is marked through reduplication of the initial mora\footnotemark , with the absolutive case left unmarked.
The following examples describe the effects of this reduplication on several word shapes, beginning with CV- and simple V-initial stems:

\footnotetext{A moraic analysis of this process is preferred to a syllabic one, as it better accounts for the vowel coalescence present in most dialects.}

\begin{columns}[cols.markup=\mutations]
  \cols malaku & \highlight{ma}malaku & cat
  \cols upe & \highlight{u'}upe & yam
  \cols eusa & \highlight{e'}eusa & nose
\end{columns}

Words beginning with \phontext{Vʔ} have this \phontext{ʔ} elided when inflected ergatively:

\begin{columns}[cols.markup=\mutations]
  \cols a'u & \highlight{a'a}u & \FIRST\SING
  \cols a'ipe & \highlight{a'a}ipe & clutter, discarded things
\end{columns}

Additionally, an epenthetic \phontext{w} is inserted if the elided portion of the stem would otherwise result in the formation of a disallowed vowel sequence:

\begin{columns}[cols.markup=\mutations]
  \cols i'a & \highlight{i'iw}a & \SECOND\SING
  \cols i'isi & \highlight{i'iw}isi & foreigner
  \cols e'afu & \highlight{e'ew}afu & dust
\end{columns}
\filbreak

\subsection{Distributives}
Nouns inflect distributively through infixation of \morphtext{me} between the stem and any ergative marking present.
As a result, \morphtext{me} surfaces as a simple prefix when the noun is in the absolutive case.

\begin{columns}[cols.markup=\mutations]
  \cols malaku & \highlight{me}malaku & cats
  \cols mamalaku & ma\highlight{me}malaku & {}
\end{columns}

\begin{columns}[cols.markup=\mutations]\label{ex:dist-syllabification}
  \cols upe & \highlight{me}upe & yams
  \cols u'upe & u\highlight{me}upe & {}
\end{columns}

In these examples, distributive nouns are rendered as plurals in the translations, as this is the \detail{closest equivalent available} without the context needed for a more accurate translation.

\labelcref{ex:dist-syllabification} provides an example of the importance of morpheme boundaries in determining correct pronunciation.
In most cases, \phomtext{əu} is realized as \phontext{o}.
However, the morpheme boundary between \morphtext{me} and the stem blocks the typical vowel coalescence from occuring.\footnote{To reflect this, these examples are syllabified as \phomtext{mə.u.pə} and \phomtext{u.mə.u.pə} respectively, rather than \phomtext{məu.pə} and \phomtext{u.məu.pə}.}

\begin{columns}[cols.markup=\mutations]\label{ex:dist-diphthong-au}
  \cols auna & \highlight{me}auna & moons
  \cols a'auna & a\highlight{me}auna & {}
\end{columns}

\begin{columns}[cols.markup=\mutations]\label{ex:dist-diphthong-eu}
  \cols eusa & \highlight{me}eusa & noses
  \cols e'eusa & e\highlight{me}eusa & {}
\end{columns}

\begin{columns}[cols.markup=\mutations]\label{ex:dist-diphthong-ai}
  \cols aihe & \highlight{me}aihe & clusters of berries
  \cols a'aihe & e\highlight{me}aihe & {}
\end{columns}

\Cref{ex:dist-diphthong-au,ex:dist-diphthong-eu,ex:dist-diphthong-ai} contrast with \labelcref{ex:dist-syllabification} in that these vowel sequences \detail{do not} cross morpheme boundaries.
As such, the diphthongs are not further reduced\footnote{\phomtext{əu} and \phomtext{əi} remain \phontext{o} and \phontext{e} respectively in dialects where this coalescence is standard.} or subjected to hiatus.
