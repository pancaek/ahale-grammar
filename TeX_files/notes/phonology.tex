\chapter{Phonology}
\section{Inventory}

\begin{table}[ht]
  \centering
  \begin{tabular}{*{7}{c}}
    \toprule
    & Labial & Alveolar & Velar & Glottal \\\midrule
    Plosive   & p      & t        & k     & ʔ       \\
    Nasal     & m      & n        &       &         \\
    Fricative & ɸ      & s        & x     & h       \\
    Sonorant  & w      & l        &       &         \\
    \bottomrule
  \end{tabular}
  \caption{Consonant inventory}
\end{table}

\begin{figure}[ht]
  \centering
  \begin{vowel}
    \vpoint{0}{3}{i}
    \vpoint{2}{3}{u}
    \vpoint{1}{1.5}{ə}
    \vpoint{1}{0}{a}
    \varrow{a}{u}
    \varrow{a}{i}
  \end{vowel}
  \caption{Vowel inventory}
\end{figure}

Diphthongs consisting of \phomtext{ə} + high vowel can be observed in a few words, however these diphthongs are prone to collapse. In all but the most conservative of dialects, these are realized as \phomtext{əi əu} \phontext{e o}.
