\chapter{Smoyds}

\begin{subexamples}
  \ex
    \gloss
      fai & NEG \\
      'ike & know \\
      pane & person \\
      ana & near\_to \\
      me & 1SG \\
      'e & REL \\
      lu & OBV \\
      kamai & send \\
      ta & PROX \\
      wu & at \\
      ipe & house \\
      fa & POSS \\
      <me>me & <PL>1 \\
    \tr You do not know the friend that will come to our house.
  \ex
    \gloss
      fai & NEG \\
      'ike & know \\
      pane & person \\
      ana & near\_to \\
      me & 1SG \\
      'e & REL \\
      <me>me & <PL>1 \\
      ati'ihi & invite \\
      wu & at \\
      ipe & house \\
  \tr You do not know the friend we're inviting (here).
  \source 5moyd \#1330
\end{subexamples}

Both of these sentences describe the same event, however they are framed with opposing volition. In (a), the friend initiates, and in (b), the family does.

In both renditions, \transtext{friend} is rendered as \nativetext{pane ana me} \transtext{person near me}.

Familial bonds in particular are especially culturally significant for speakers of \langname . Coupled with the tendency for entire families to live together, and to only move out and part ways to start one's own family, the concept of closeness has extended into a linguistic metaphor for emotional attachment.

(a) uses the reflexive form of \nativetext{kamai}, emphasizing that this was done without prompting from the family (that is to say, this was previously unannounced and unknown). This contrasts with \nativetext{heemi} \transtext{to come}, which implies a relationship to the destination.

\nativetext{Lu} in (a) serves as a resumptive pronoun, allowing the reflexive reading of \nativetext{kamai ta}. This is necessary because \nativetext{pane ana me} is within a relative clause, and would otherwise leave \nativetext{ta} without a referent.

Also, \nativetext{wu} (in both cases) is being used to denote location, rather than performing its primary function conveying \transtext{on top}.

\begin{example}
  \gloss
    mai & time \\
    eta & age \\
    -'e & ASSOC \\
    pa & now \\
    pa\allo pami & ERG\allo marble \\
    ahi & fasten \\
    eusa & nose \\
  \tr When I was a child, I got a marble stuck in my nostril.
  \source 5moyd \#1289
\end{example}

\nativetext{Mai eta'e} must combine with the adverb \nativetext{pa} to form a topic-like adverbial phrase. This can be thought of as a phrase \nativetext{mai... pa} used to introduce a timeframe for the events following. Here, that frame is \nativetext{eta'e} \transtext{young age}, however this could be any number of things. One may expect a relative clause here, however, such a construction cannot be used for the purposes of temporal framing. Relative clauses are too syntactically heavy to fit this construction, and are generally \notabletext{determinate}.

One final note is that, as with other body parts, \nativetext{eusa} is left unmarked for possession when it is easily inferred (in most cases as the speaker when no other context is given).


\begin{example}
  \gloss
    <me>me & <ERG>1SG \\
    kisufi & CERT \\
    keke & eat \\
    neulina & stew \\
    tu'ase & wheat \\
    -la & ADJZ \\
    \tr I intend to eat oatmeal.
    \source 5moyd \#1252
  \end{example}

This is fairly standard, although the use of a certainty particle in place of \transtext{intend to} is more ambiguous than this translation makes it out to be. It could also be read as \transtext{I'll probably eat oatmeal.}, which could also be apt. In this situation, I've chosen to translate it as the first, meaning that we aren't sure whether some external factor will stop this from happening.

\begin{example}
  \gloss
    <me>me & <ERG>1SG \\
    i- & PST.IPFV \\
    ka & open \\
    tepa & container \\
    pa & now \\
    siha & happen \\
    tele & part \\
    -la & ADJZ \\
  \tr I was opening the bottle and it shattered.
  \source 5moyd \#1228
\end{example}

Firstly, the choice to translate \transtext{bottle} with \nativetext{tepa} rather than a more specific word is because it is rare for drinking containers to be lidded, and so opening them wouldn't make sense.

In translating the verbs, it makes more sense to draw attention to the change of state of the bottle. Rather than implying the shattering is caused by the original agent (as using a particular verb for shattering would), the construction \nativetext{siha X-la} assumes that this happened on account of an external(/unintroduced) force, where \nativetext{-la} describes the end state. \nativetext{Pa} serves to link these clauses, and so rather than an unknown reading, we get a causative reading which is framed against the event rather than its original agent.

\begin{example}
  \gloss
    sulau & open \\
    tanu & door \\
    ta & PROX \\
    -'e & ASSOC \\
    kate & simplicity \\
  \tr The door opens easily.
  \alt To open this door is easy.
  \source 5moyd \#1395
\end{example}

The core argument of the verb phrase is deemphasized by being placed after the verb rather than before (as is typical with indicative phrases). This could also be interpreted as an interrogative, disambiguated solely through prosody. In very formal speech, deemphasized core arguments will be marked with the obviate, but this marking is usually unnecessary unless the deemphasis is particularly important or used to form contrast with another concept.

\nativetext{Ta} functions as a sort of topic marker, where \nativetext{kate} modifies the newly established topic. This strategy is common in forming simple adverbials.

\begin{example}
  \preamble wa'u lai eteuteukeu.
  \gloss
    wa'u & cheese \\
    lai & ongoing \\
    e- & NPST.PFV \\
    teuteukeu & rot \\
  \tr The cheeses have deteriorated. (and continue to do so further)
  \source 5moyd \#1468
\end{example}

This example uses the nonpast perfective, even though it describes a state that had been reached in the past. This signifies that the change is recent, or has just been observed. This is reinforced by the use of the temporal adverb \nativetext{lai}. Another combination may have implied that this change of state had been occurring in the past, but had stopped after some duration. For example:

\begin{example}
  \preamble pane ne'e maila.
  \gloss
    pane & person \\
    ne'e & hour \\
    m- & PST.PFV \\
    aila & scream \\
  \tr The person screamed for an hour. (but they've stopped now)
\end{example}

As this second example shows us, this adverb can either be a proper temporal adverb, or simply a duration over which an event has occurred.
