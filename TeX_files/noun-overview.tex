\chapter{Nominals}

\section{Inflectional morphology}

Inflectional morphology is rather limited, as nouns may inflect for one of two cases (ergative or absolutive), and may be either singulative or collective. The ergative is marked with reduplication of the initial syllable. When a collective noun is in the ergative case, the number marking is infixed between the stem and the reduplicated segment.

\begin{columns}[cols.markup=\mutations]
  \cols \textbf{tai} & \highlight{tai}tai & \ERG\allo container
  \cols {} & \highlight{me}tai & 〈\COLL〉container
  \cols {} & \highlight{taime}tai & \ERG\allo〈\COLL〉container
\end{columns}

\begin{columns}[cols.markup=\mutations]
  \cols \textbf{aihe} & \highlight{ai}aihe & \ERG\allo rubbish
  \cols {} & \highlight{me}aihe & 〈\COLL〉rubbish
  \cols {} & \highlight{aime}aihe & \ERG\allo〈\COLL〉rubbish
\end{columns}

\begin{columns}[cols.markup=\mutations]
  \cols \textbf{eusa} & \highlight{e'}eusa & \ERG\allo ear
  \cols {} & \highlight{me}eusa & 〈\COLL〉ear
  \cols {} & \highlight{eme'}eusa & \ERG\allo〈\COLL〉ear
\end{columns}

Note the syllabification of \rom{eusa}, only the \rom{e} is reduplicated.

\section{Direct-inverse alignment}
\langname{}\ has direct-inverse alignment, which means that the voice of the verb is dependent on the constituents position along a ``person hierarchy''. The \highlight{direct voice} is used when the agent is higher in the hierarchy from the patient, and \highlight{inverse voice} for the opposite. In a typical direct-inverse system without case marking, a proximate/obviative distinction is needed to distinguish the roles of two 3rd person verbal arguments. \langname{} still retains these morphemes, but their usage has shifted since case marking is used to convey similar information. Only inverse voice is marked overtly, with the suffix \rom{-si}.


\begin{example}
  \gloss
  ha^ & ERG \\
  hawi & bunny \\
  awale^ & AUG \\
  awale & touch \\
  -si & INV \\
  a'u & 1.SING \\
  \tr The bunny nuzzles me.
\end{example}



The antiquated proximate and obviate primarily serve to reinforce role marking. Patients may be fronted and marked with obviative \rom{lu} to indicate a focus.

\begin{example}
  \gloss
  lu & OBV \\
  a'u & 1.SING \\
  ha^ & ERG \\
  hawi & bunny \\
  awale^ & AUG \\
  awale & touch \\
  -si & INV \\
  \tr The bunny nuzzles \underline{me}.
\end{example}

Agents may be de-emphasized via an adjacent mechanism, which creates something of similar function to a passive voice.

\begin{example}\label{ex:bunny-decl}
  \gloss
  awale^ & AUG \\
  awale & touch \\
  -si & INV \\
  ta & PROX \\
  ha^ & ERG \\
  hawi & bunny \\
  a'u & 1.SING \\
  \tr I am nuzzled by the bunny.
\end{example}


\detail{Careful! Fronted verbs in a clause without role marking indicate the interrogative.}

\begin{example}
  \gloss
  awale^ & AUG \\
  awale & touch \\
  -si & INV \\
  ha^ & ERG \\
  hawi & bunny \\
  a'u & 1.SING \\
  \tr The bunny nuzzled me?.
\end{example}

To show focus within interrogatives, the same role marking particles can be used, but the movement works differently; only the focus of the patient actually causes movement.\marginpar{This is because the interrogative movement happens before the focus movement, and there is nowhere higher up free to hold an agent. On the other hand, patients can still be partially raised.}

\begin{example}
  \gloss
  awale^ & AUG \\
  awale & touch \\
  -si & INV \\
  lu & OBV \\
  a'u & 1.SING \\
  ha^ & ERG \\
  hawi & bunny \\
  \tr The bunny nuzzled \underline{me}?.
\end{example}

\begin{example}\label{ex:bunny-int}
  \gloss
  awale^ & AUG \\
  awale & touch \\
  -si & INV \\
  ta & PROX \\
  ha^ & ERG \\
  hawi & bunny \\
  a'u & 1.SING \\
  \tr \underline{The bunny} nuzzled me?.
\end{example}

Note that the word orders of \cref{ex:bunny-decl,ex:bunny-int} are the same. The interrogative results from raising of the verb, while the agent de-emphasis is through lowering. In speech, the interrogatives also have a different intonation contour, which helps with additional disambiguation.

Across clauses, \rom{lu} and \rom{ta} function as resumptive pronouns:

\begin{example}
  \gloss
  hawi & bunny \\
  wa, & AFF \\
  au^ & ERG \\
  au & 1.SING \\
  awale^ & AUG \\
  awale & touch \\
  lu & OBV \\
  \tr A bunny! I pet it.
\end{example}
