\chapter{Phonology}

Generally speaking, \langname{} has 12 consonants and 4 vowels. The standard phonlogical inventory can be arranged like this:

\begin{table}[ht]


  \centering
  \begin{tabular}{*{7}{c}}
    \toprule
              & Labial & Alveolar & Velar & Glottal \\\midrule
    Plosive   & p      & t        & k     & ʔ       \\
    Nasal     & m      & n        &       &         \\
    Fricative & f      & s        & x     & h       \\
    Sonorant  & w      & l        &       &         \\
    \bottomrule
  \end{tabular}
  \caption{Consonant inventory}
\end{table}

This is not necessarily an incorrect analysis, but it does leave out a few notable details. The consonant written here as \phomtext{f} is realized as \phontext{ɸ}, except word--initially. One could argue that the underlying form is \phomtext{ɸ}. In that case, we would posit a rule \phonl{ɸ}{f}{\#}.

\detail{To be honest, I'm not sure which analysis is better. The instability of \phontext{ɸ} leads me to belive that it will probably not remain \phontext{ɸ} in such a broad environment for long though.}

\phomtext{l} is also a bit strange, as I've encountered \phontext{ɭ} on several occasions, with no other retroflexes evident.

\begin{figure}[htbp]
  \centering
  \begin{vowel}
    \vpoint{0}{3}{i}
    \vpoint{2}{3}{u}
    \vpoint{1}{1.5}{ə}
    \vpoint{1}{0}{a}
    \varrow{a}{u}
    \varrow{a}{i}
  \end{vowel}
  \caption{Vowel inventory}
\end{figure}
