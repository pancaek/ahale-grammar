\chapter{Phonology}

Generally speaking, \langname{} has 12 consonants and 4 vowels. The standard phonlogical inventory can be arranged like this:

\begin{table}[ht]


  \centering
  \begin{tabular}{*{7}{c}}
    \toprule
              & Labial & Alveolar & Velar & Glottal \\\midrule
    Plosive   & p      & t        & k     & ʔ \orthotext{'}      \\
    Nasal     & m      & n        &       &         \\
    Fricative & f      & s        & x     & h       \\
    Sonorant  & w      & l        &       &         \\
    \bottomrule
  \end{tabular}
  \caption{Consonant inventory}
\end{table}

In addition to the modal voice, \langname{} also distinguishes a breathy voice for all consonants. The romanized form of breathy consonants is the base grapheme followed by \orthotext{h}.

\begin{figure}[htbp]
  \centering
  \begin{vowel}
    \vpoint{0}{3}{i}
    \vpoint{2}{3}{u}
    \vpoint{1}{1.5}{ə}
    \vpoint{1}{0}{a}
    \vpoint{.3}{2.7}{\footnotesize(ai)}
    \vpoint{1.7}{2.7}{\footnotesize(au)}
    \varrow{a}{u}
    \varrow{a}{i}
  \end{vowel}
  \caption{Vowel inventory}
\end{figure}

Phonotactics (C)V, where V also includes diphthongs. Eg.\ \phomtext{tau} is a valid syllable, but \phomtext{teu} necessitates a second syllable, \phontext{tə.u}.