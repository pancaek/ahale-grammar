
\chapter{Discourse}

\section{Repair Questions}
As an example, lets imagine a hypothetical speaker just said the following:
\ex
\begingl
\glpreamble hahawi keke lu ɸumau
\pronounced{ˈɣa.ɣa.wi ˈke.kə ˈlu ˈɸu.mau}\endpreamble
ha\textasciitilde hawi[\textsc{erg\textasciitilde}rabbit]
keke[\textsc{npst.ipfv-}eat]
lu[\textsc{obv}]
ɸumau[\textsc{abs,}grass]
\glft `The rabbit (as opposed to something else) is eating the grass.'
\endgl
\xe

\subsection{Nominals}

If someone mishears, or for whatever reason needs clarification on the arguments of a transitive verb, the obviative and proximate markers can be used.

\begin{paracol}{2}
If the listener only hears \langword{hahawi ke\-ke lu}, and not the \textit{patient}, the listener can ask the following:
\ex
\begingl
\glpreamble lu?
\pronounced{ˈlu}\endpreamble
lu[\textsc{obv}]
\glft `Eating what?'
\endgl
\xe
\switchcolumn

If a listener only hears \langword{keke lu ɸumau,} and not the \textit{agent}, the listener can ask the following:

\ex
\begingl
\glpreamble ta?
\pronounced{ˈta}\endpreamble
ta[\textsc{prox}]
\glft `What is eating grass?'
\endgl
\xe
\end{paracol}

\subsection{Verbal}
If our listener only heard \langword{``hahawi --- lu ɸumau'',} the response may be:
\ex
\begingl
\glpreamble ta sihu lu?
\pronounced{ˈta ˈsi.hu ˈlu}\endpreamble
ta[\textsc{prox}]
sihu[happen]
lu[\textsc{obv}]
\glft `The rabbit is doing what to grass?'
\endgl
\xe

\langword{ta} and \langword{lu} are used here in a resumptive fashion, rather than repeating the content words. This implies more confidence, in that repeating \langword{hahawi} or \langword{ɸumau} may imply that the listener is also unsure of these components as well, rather than just the verb.

Because this phrase is somewhat of a standard one, it is shortened in colloquial speech. The most aggresive of these shortenings being [ˈtasul(ə)].

\section{Responding to Repair Questions}\label{sec:repair_response}

Repair questions can be responded to quite similarly to how a ``standard'' question would be. The main difference is the necessity of the associative particle, \langword{ʔe} to connect \langword{ta} or \langword{lu} to the appropriate content word(s). For example:

\ex
\begingl
\glpreamble ta ʔehawi
\pronounced{ˈta ˈʔe.ha.wi}\endpreamble
ta[\textsc{prox}]
ʔe[\textsc{assoc}]
hawi[rabbit]
\glft `The rabbit (is eating grass).'
\endgl
\xe
\section{Self-Correction}
\subsection{In conjunction with \langword{ʔe} Ellipsis}

The same structures used to respond to repair questions may be employed to correct or clarify the ellipsed NP, for example if the remaining information is too ambiguous, incorrect, or is simply no longer relevant.

\ex<adjective_correction>
\begingl
\glpreamble tatakaʔe ala ta ʔekatu kulasi me
\pronounced{ˈta.ta.ka.ʔə ˈa.la ˈta ˈʔe.ka.tu ˈku.la.si ˈme}\endpreamble
tataka[\textsc{erg:}rock]
ʔe[\textsc{assoc}]
ala[white]
ta[\textsc{prox}]
ʔe-katu[ʔe-sharpness]
kulasi[hurt\textsc{:inv}]
me[\textsc{abs:1sg}]
\glft `The white --- no, sharp --- rock is hurting me.'
\endgl
\xe

When \langword{ʔe} is used in discourse repair, it attaches to the correction, rather than to \langword{ta} or \langword{lu.} This the the opposite of when \langword{ʔe} is used associatively (see \Sref{ch:adjectives}), where \langword{ʔe} attaches to the noun being described.

%TODO: Self-repaired nouns and verbs

\subsection{Noun Repair}
\ex<adjective_correction>
\begingl
\glpreamble hahawi ta ʔetatakaʔe ala kulasi me
\pronounced{ˈɣa.ha.wi ta ˈʔe.ta.ta.ka.ʔə ˈa.la ˈku.la.si ˈme}\endpreamble
hahawi[\textsc{erg:}rabbit]
ta[\textsc{prox}]
ʔe[\textsc{assoc}]
tataka[\textsc{erg:}rock]
ʔe[\textsc{assoc}]
ala[white]
kulasi[hurt\textsc{:inv}]
me[\textsc{abs:1sg}]
\glft `The white rabbit --- no, rock ---  is hurting me.'
\endgl
\xe

The patient can be repaired the same way, though using \langword{lu} instead of \langword{ta.}
\subsection{New Adjectives}
Similarly, if a noun which previously had no adjectives modifying it needs to have one added in the middle of discourse, the following can be done:

\ex
\begingl
\glpreamble ta nene xase lu ʔe kela
\pronounced{ˈta ˈne.nə ˈxa.sə ˈlu.ʔə ˈke.la}\endpreamble
ta[\textsc{prox}]
nene[paint]
xase[spill]
lu[\textsc{obv}]
ʔe[\textsc{assoc}]
kela[green]
\glft `The paint spilled, it's green.'
\endgl
\xe

%? Does green exist as its own basic term? Decide.

Special attention should be given when referencing arguments of labile verbs (see \Sref{sec:labile_verbs}). While \langword{ta} marks \langword{nene} as the most agent-like, it is still morphologically and syntactically the verb's patient (note the lack of ergative marking). Because of this, core (S) arguments of labile verbs should be used with the resumptive pronoun used with patients, \langword{lu.}

Also, adjectival additions such as this are strongly preferred to be done \textit{after} the verb, particularly with intransitive verbs. These types of corrections are done differently in that the adjective \textit{is} associated to \langword{lu}. This is different from correcting the use of an incorrect adjective, seen in Example \getfullref{adjective_correction}.

\section{Affirmative/Negative Particles}

\langname\ has an extensive system of discouse particles, particularly those relating to certainty, and various levels of agreement. In most senarios, these are simply placed before the verb phrase, but a select few have special clitic forms, which attach directly to the verb.

\subsection{Certainty}

\langname 's `certainty particles', as they will be referred to from now on, exist on a continuum, ranging from \langword{naʔu} `unwavering certainty of falsehood' to \langword{tamuka} `unwavering certainty of truthfullness', with medians such as \langword{tesune} `firm skepticism', and \langword{ewili} `easily swayed uncertainty'.

A more comprehensive list is as follows: \langword{naʔu, lelewine, tesune, ewili, kisuɸi, mili, tamuka.} The undescribed forms fall somewhere within the space in the continuum, such that their definitions are solely defined by their relationships to each other.
\subsubsection{\langword{Lelewine}}
For example, if a businessperson of questionable legitimacy approaches you with a suspiciously low price on an expensive item, one could reply using these particles to show (dis)trust.

\ex
\begingl
\glpreamble lelewine i naʔa lu me memalaku ilaʔe
\pronounced{ˈle.lə.wi.nə ˈi ˈna.ʔa ˈlu ˈme ˈme.ma.la.ku ˈi.la.ʔə}\endpreamble
lelewine[\textsc{uncert}]
i[inside]
naʔa[\textsc{npst.ipfv:}sell]
lu[\textsc{obv}]
me[\textsc{1sg}]
memalaku[pl:cat]
ila\footnotemark[wealth]
ʔe[\textsc{assoc}]
\glft `You are selling me cats, for only this much (surely this isn't right, but just maybe)?'
\endgl
\xe

Interrogative meaning is additionally conveyed prosodically, which is particularly important in that ditransitives have fixed V2 order.

\footnotetext{`Wealthy cats' are not particularly sensible, and therefore this associative is used to refer to the cost of them instead. This happens often when an adjective is not prototypically used to describe something -- its meaning in this context shifts slightly.}

\ex<uncertain-clitic>
\begingl
\glpreamble i ewi-naʔa lu me memalaku ilaʔe
\pronounced{ˈi ˈe.wi.na.ʔa ˈlu ˈme ˈme.ma.la.ku ˈi.la.ʔə}\endpreamble
i[inside]
ewi-naʔa[\textsc{uncert-npst.ipfv:}sell]
lu[\textsc{obv}]
me[\textsc{1sg}]
memalaku[pl:cat]
ila[wealth]
ʔe[\textsc{assoc}]
\glft `You are selling me cats, for only this much (surely this isn't right, but just maybe)?'
\endgl
\xe

The particle \langword{lelewine} can be shortened and attached (orthographically as well) as in the above, to have the same meaning, but trading a small amount of clarity for a stylistic choice of words. The shortened, attached forms are also preferred (if they exist), when a verb is used on it's own as another form of echo question.

% \begin{tikzpicture}
%   \contour[contour raise=0.5cm]
%   {|[10]Where |[3]are you |[6]go|[1]ing?|[0]}
% \end{tikzpicture}

\ex
\begingl
\glpreamble imipi ɸai luʔe paki?
\pronounced{ˈi.mi.pi ˈɸai ˈlu.ʔə ˈpa.ki}\endpreamble
imipi[\textsc{pst.ipfv}:think]
ɸai[\textsc{neg}]
lu[\textsc{obv}]
ʔe[\textsc{assoc}]
paki[truth]
\glft `You weren't thinking of me truthfully?'
\endgl
\xe

Note the use of the imperfective here, this is important because it shows the speaker still knows they have not swayed the other person. To use the perfective would, in this senario, give the impression of an implied `but now you have', which is not the case here.

In response to Example \getfullref{uncertain-clitic}, a still uncertain buyer may affirm previous thoughts with one of the following:

\ex
\begingl
\glpreamble ewi-naʔa
\pronounced{ə.wiˈna.ʔa}\endpreamble
i[inside]
ewi-naʔa[\textsc{uncert-npst.ipfv:}sell]
lu[\textsc{obv}]
me[\textsc{1sg}]
memalaku[pl:cat]
ila[wealth]
ʔe[\textsc{assoc}]
\glft `You're selling? I don't know about this...'
\endgl
\xe

Alternatively, one can simply repeat (or choose another) certainty particle to use in isolatio. When used this way, the full form is mandatory.

\ex
\begingl
\glpreamble lelewine...
\pronounced{ ˈle.lə.wi.nə}\endpreamble
lelewine[\textsc{uncert}]
\glft `I'm not so sure about this...'
\endgl
\xe