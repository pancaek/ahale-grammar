\documentclass[12pt, a4paper]{pancake-book}
\usepackage{fontspec}
\let\ordinal\relax %! Fixes \ordinal conflict with memoir
\usepackage[us]{datetime}
\newdate{startdate}{15}{3}{2020}
\usepackage{paracol}
\usepackage{enumitem}
\setlist[itemize]{topsep=0pt, itemsep=.5\itemsep}

\usepackage{expex, tikzvowel, phonrule}
\input{tikztone.tex}

\usepackage[hidelinks]{hyperref}
\hypersetup{linktoc=all}

\titlingpageend{\clearforchapter}{\clearforchapter} %! Titlingpage/openany fix for blank page
\pagestyle{ruled}
\makeoddfoot{plain}{}{}{\thepage}
\makeevenfoot{plain}{\thepage}{}{}
\copypagestyle{title}{empty}

\setmainfont{Libertinus Serif}

\setlength{\parindent}{0pt}
\nonzeroparskip
\renewcommand\cftdotsep{2.5}

\lingset{
  glstyle=nlevel,
  belowglpreambleskip={-.5\parskip},
  aboveglftskip={-.5\parskip},
  exskip={0pt},
  glneveryline={,\it,}
}

% Doc specific stuff
\newcommand{\langword}[1]{\textit{#1}}
\newcommand{\langname}{Ahale}

\newcommand{\pronounced}[1]{\vskip-0.5\parskip[#1]}
\makeglossaries

\begin{document}
\title{\langname : A Complete Reference}
\author{Pancake}
\date{\DTMusedate{startdate} -- \DTMtoday}
\frontmatter

\maketitle
\tableofcontents

\printglossaries
\mainmatter
\chapter{Introduction}
\section{Document structure}
\subsection{Prose}
All prose within the main block of text is self-contained, such that reading it on its own is sufficient to a gain basic understanding of the material.
Other details which accompany the main text support and expand its content from several different prespectives.
The additional materials are designed this way not to discourage deeper reading, but rather to keep the most important details easily accessible without sacrificing depth of coverage for nuanced topics.

\subsection{Margin notes}
\subsection{Footnotes}
\chapter{Phonology}
\section{Inventory}

\begin{table}[ht]
  \centering
  \begin{tabular}{*{7}{c}}
    \toprule
    & Labial & Alveolar & Velar & Glottal \\\midrule
    Plosive   & p      & t        & k     & ʔ       \\
    Nasal     & m      & n        &       &         \\
    Fricative & ɸ      & s        & x     & h       \\
    Sonorant  & w      & l        &       &         \\
    \bottomrule
  \end{tabular}
  \caption{Consonant inventory}
\end{table}

\begin{figure}[ht]
  \centering
  \begin{vowel}
    \vpoint{0}{3}{i}
    \vpoint{2}{3}{u}
    \vpoint{1}{1.5}{ə}
    \vpoint{1}{0}{a}
    \varrow{a}{u}
    \varrow{a}{i}
  \end{vowel}
  \caption{Vowel inventory}
\end{figure}

Diphthongs consisting of \phomtext{ə} + high vowel can be observed in a few words, however these diphthongs are prone to collapse. In all but the most conservative of dialects, these are realized as \phomtext{əi əu} \phontext{e o}.

\chapter{The noun phrase}
\section{Inflection and morphophonology}
\subsection{Case}
\langname\ has only two cases: the ergative and the absolutive.
The ergative case is marked through reduplication of the initial mora\footnotemark , with the absolutive case left unmarked.
The following examples describe the effects of this reduplication on several word shapes, beginning with CV- and simple V-initial stems:

\footnotetext{A moraic analysis of this process is preferred to a syllabic one, as it better accounts for the vowel coalescence present in most dialects.}

\begin{columns}[cols.markup=\mutations]
  \cols malaku & \highlight{ma}malaku & cat
  \cols upe & \highlight{u'}upe & yam
  \cols eusa & \highlight{e'}eusa & nose
\end{columns}

Words beginning with \phontext{Vʔ} have this \phontext{ʔ} elided when inflected ergatively:

\begin{columns}[cols.markup=\mutations]
  \cols a'u & \highlight{a'a}u & \FIRST\SING
  \cols a'ipe & \highlight{a'a}ipe & clutter, discarded things
\end{columns}

Additionally, an epenthetic \phontext{w} is inserted if the elided portion of the stem would otherwise result in the formation of a disallowed vowel sequence:

\begin{columns}[cols.markup=\mutations]
  \cols i'a & \highlight{i'iw}a & \SECOND\SING
  \cols i'isi & \highlight{i'iw}isi & foreigner
  \cols e'afu & \highlight{e'ew}afu & dust
\end{columns}
\filbreak

\subsection{Distributives}
Nouns inflect distributively through infixation of \morphtext{me} between the stem and any ergative marking present.
As a result, \morphtext{me} surfaces as a simple prefix when the noun is in the absolutive case.

\begin{columns}[cols.markup=\mutations]
  \cols malaku & \highlight{me}malaku & cats
  \cols mamalaku & ma\highlight{me}malaku & {}
\end{columns}

\begin{columns}[cols.markup=\mutations]\label{ex:dist-syllabification}
  \cols upe & \highlight{me}upe & yams
  \cols u'upe & u\highlight{me}upe & {}
\end{columns}

In these examples, distributive nouns are rendered as plurals in the translations, as this is the \detail{closest equivalent available} without the context needed for a more accurate translation.

\labelcref{ex:dist-syllabification} provides an example of the importance of morpheme boundaries in determining correct pronunciation.
In most cases, \phomtext{əu} is realized as \phontext{o}.
However, the morpheme boundary between \morphtext{me} and the stem blocks the typical vowel coalescence from occuring.\footnote{To reflect this, these examples are syllabified as \phomtext{mə.u.pə} and \phomtext{u.mə.u.pə} respectively, rather than \phomtext{məu.pə} and \phomtext{u.məu.pə}.}

\begin{columns}[cols.markup=\mutations]\label{ex:dist-diphthong-au}
  \cols auna & \highlight{me}auna & moons
  \cols a'auna & a\highlight{me}auna & {}
\end{columns}

\begin{columns}[cols.markup=\mutations]\label{ex:dist-diphthong-eu}
  \cols eusa & \highlight{me}eusa & noses
  \cols e'eusa & e\highlight{me}eusa & {}
\end{columns}

\begin{columns}[cols.markup=\mutations]\label{ex:dist-diphthong-ai}
  \cols aihe & \highlight{me}aihe & clusters of berries
  \cols a'aihe & e\highlight{me}aihe & {}
\end{columns}

\Cref{ex:dist-diphthong-au,ex:dist-diphthong-eu,ex:dist-diphthong-ai} contrast with \labelcref{ex:dist-syllabification} in that these vowel sequences \detail{do not} cross morpheme boundaries.
As such, the diphthongs are not further reduced\footnote{\phomtext{əu} and \phomtext{əi} remain \phontext{o} and \phontext{e} respectively in dialects where this coalescence is standard.} or subjected to hiatus.

\filbreak
\section{Distributives}
Rather than describing nouns by their number, \langname\ describes nouns as either distributive or non-distributive.
Uses of the distributive form can be roughly divided into the following categories:

\begin{itemize}
  \item To form collective plurals
  \item To form plurals
  \item For prototypical distributive use
  \item To make generalizations
\end{itemize}

\subsection{Collective vs.\ collective plural}
Inherently dual nouns and collective nouns \nativeparen{物有二} may be modified by \DST\ to refer to multiple \detail{complete} sets.

\begin{examples}
  \baarucols{2}
  \ex
    \script 手
    \romanization heu
    \gloss
      ∅- & NDST \\
      heu & hand \\
    \tr (Both) hands
    \not (Two) hands
  \ex
    \script 各手
    \romanization meheu
    \gloss
      <me>heu & <DST>hand \\
    \tr Sets of hands
\end{examples}

\subsection{As a prototypical distributive}
\begin{example}\label{ex:dst-pred-cop}
  \script 各体鳥立垢
  \romanization menula sipu ta feisa
  \gloss
    <me>nula & <DST>body \\
    sipu & bird \\
    ta & ta \\
    feisa & filth \\
  \tr These bird carcasses (of many types) are so gross.
\end{example}

This sentence may appear to be a generalization of the qualities of bird carcasses.
However, generalized readings are more common with verbs inherently descriptive of opinion, rather than the predicational copular phrase \labelcref{ex:dst-pred-cop} presents.
\Cref{ex:dst-pred-cop} also acknowledges that under different circumstances, the bird carcasses \detail{may be perceived differently}.

\begin{example}\label{ex:ndst-pred-cop}
  \script 体鳥立垢
  \romanization nula sipu ta feisa
  \gloss
    nula & body \\
    sipu & bird \\
    ta & ta \\
    feisa & filth \\
  \tr These bird carcasses are so gross.
\end{example}

Being the non-distributive form of \labelcref{ex:dst-pred-cop}, \cref{ex:ndst-pred-cop} refers to a single type of carcass. Number is unspecified, instead referring to \detail{specific referents} established with prior context.

\subsection{Making generalizations/expressing opinions}
As mentioned previously, the use of \morphtext{me} to express opinion is mostly confined to its use alongside verbs describing emotion or state of mind.

\begin{examples}
  \ex
    \romanization a'au a'aeu kita'e auka
    \gloss
      a'\allo & ERG \\
      au & 1 \\
      a'aeu & enjoy\_presense \\
      ∅- & NDST \\
      kita & tree \\
      -'e & ASSOC \\
      auka & oak \\
    \tr I like those oak trees.
  \ex\label{ex:generalization-trees}
    \romanization a'au a'aeu mekita'e auka
    \gloss
      a'\allo & ERG \\
      au & 1 \\
      a'aeu & enjoy\_presense \\
      <me>kita & <DST>tree \\
      -'e & ASSOC \\
      auka & oak \\
    \tr I like oak trees.
\end{examples}

Generalized statements are often made by omitting personal pronouns, and optionally emphasizing the topic of the opinion.

In situations such as \labelcref{ex:generalization-trees} where the agent is making the generalization, the process of emphasizing the opinion is quite simple, because the patient can simply be fronted:

\begin{example}
  \romanization mekita'e auka a'aeu
  \gloss
    <me>kita & <DST>tree \\
    -'e & ASSOC \\
    auka & oak \\
    a'aeu & enjoy\_presense \\
  \tr Oak trees are pleasant.
\end{example}
\filbreak

This changes slightly when the agent is the opinion's topic. If further fronting cannot occur, \native{立} is used to emphatically mark agency:

\begin{example}
  \romanization ta simesikama aseisi
  \gloss
    ta & ta \\
    si\allo & ERG\\
    <me>sikama & <DST>buzzing\_insect \\
    asei & scare \\
    -si & INV \\
  \tr Insects are scary.
  \lit Insects scare (me).
\end{example}

An exception to the verbal restrictions surrounding generalization occurs when expressing opinions about the \detail{attributes} of an object.
\aside{Qualities which can be described this way are idealized or culturally significant.
Opinions on these qualities are seen as more important, and the choice to describe them this way gives the opinion more weight than otherwise imparted by a more peripherastic construction.}
One caveat is that the attribute must be one which can be described with the associative particle, \rom{'e}.

\begin{example}
  \script 各木亜脊立煩
  \romanization mekita'e pala ta sika
  \gloss
    <me>kita & <DST>tree \\
    -'e & ASSOC \\
    pala & height \\
    ta & ta \\
    sika & worry \\
  \tr Tall trees are foreboding.
\end{example}
\subsection{Singular vs.\ plural}
The plural reading of \morphtext{me} occurs with personal pronouns, foreign loans that are not \native{物有二}, or loans which have not been sufficiently nativized, as with \labelcref{ex:plural-pronouns}.
Established loans, such as \rom{i'isi} \trans{foreigner} follow the prototypical distibutive reading.

\begin{columns}[cols.markup=\mutations]\label{ex:plural-pronouns}
  \cols leunu & meleunu & cactus/cactuses & (Lérvima \rom{lonón}) \\
  \cols neke & meneke & goat/goats & (Epe \rom{ŋehke})
\end{columns}


\section{Compound nominals}
Generally speaking, multiple NPs will not follow each other in default position, and will be marked with \native{立/呂} when focus moves NPs adjacent to each other.
Compound nominals can be formed through intentional adjacency of nouns (forming \highlight{adjacency compounds}), which can be seen in \cref{ex:ndst-pred-cop}. In this situation, the first noun is treated as the head, while the following nouns progressively narrow the meaning.
In the case of \cref{ex:ndst-pred-cop} this is roughly analogous to implicit `of' between the component nouns (body [of] bird), but the use is wider in \langname\ than it is in English.

Namely, to narrow or shift meanings by chaining near synonyms or semantically related concepts:
\begin{example}
  \romanization wasi nauwa
  \gloss
    wasi & river\_stone \\
    nauwa & flowing\_water \\
  \tr stones (from the riverbed)
\end{example}

\begin{example}\label{ex:adjacency-chain}
  \romanization wasi nauwa ileu'ei
  \gloss
    wasi & river\_stone \\
    nauwa & flowing\_water \\
    ileu'ei & origin \\
  \tr stones (from the mouth of the river)
  \alt waystones (which mark the river)
\end{example}

\Cref{ex:adjacency-chain} shows an interesting feature of adjacency compounds, this being that the relationship between dependent nouns is not necessarily linear. The primary translation results from considering these nested compounds:

\unskip
\begin{tightcenter}
  \large [wasi [nauwa [ileu'ei]]]
\end{tightcenter}

and the alternative translation is a result of considering both dependent nouns as modifying the same head:


\begin{tightcenter}
  \large [wasi [nauwa] [ileu'ei]]
\end{tightcenter}

\begin{example}
  \romanization wasi nauwa (ta'e) ti'u
  \gloss
    wasi & river\_stone \\
    nauwa & flowing\_water \\
    ta & ta \\
    -'e & ASSOC \\
    ti'u & unevenness\footnotemark \\
  \tr pointed stones (from the riverbed)
\end{example}

\footnotetext{\rom{Ti'u} specifically refers to unevenness which favors a particular side over another. Teardrop shapes are perhaps a good example to distinguish from unevenness built upon inconsistency. Teardrop shapes can be described with \rom{ti'u} because of the contrast of the pointed end to the smooth bulb. Long and flat rocks may also be described this way, particularly if they are oblong. Unevenness as related to texture or consistency is translated instead through \rom{hitexu}, an ideophone which mimics various types of scraping noises.}





\end{document}