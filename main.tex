\documentclass[openany, 12pt, b5paper]{memoir}
\usepackage{fontspec}
\let\ordinal\relax %! Fixes \ordinal conflict with memoir
\usepackage[us]{datetime}
\newdate{startdate}{15}{3}{2020}
\usepackage{paracol}
\usepackage{enumitem}
\setlist[itemize]{topsep=0pt, itemsep=.5\itemsep}

\usepackage{expex, tikzvowel, phonrule}
\input{tikztone.tex}

\usepackage[hidelinks]{hyperref}
\hypersetup{linktoc=all}

\titlingpageend{\clearforchapter}{\clearforchapter} %! Titlingpage/openany fix for blank page
\pagestyle{ruled}
\makeoddfoot{plain}{}{}{\thepage}
\makeevenfoot{plain}{\thepage}{}{}
\copypagestyle{title}{empty}

\setmainfont{Libertinus Serif}

\setlength{\parindent}{0pt}
\nonzeroparskip
\renewcommand\cftdotsep{2.5}

\lingset{
  glstyle=nlevel,
  belowglpreambleskip={-.5\parskip},
  aboveglftskip={-.5\parskip},
  exskip={0pt},
  glneveryline={,\it,}
}

% Doc specific stuff
\newcommand{\langword}[1]{\textit{#1}}
\newcommand{\langname}{Ahale}

\newcommand{\pronounced}[1]{\vskip-0.5\parskip[#1]}
\begin{document}
\title{\langname : A Complete Reference}
\author{Pancake}
\date{\displaydate{startdate} - \today}
\frontmatter
\begin{titlingpage}
  \maketitle
\end{titlingpage}

\begin{KeepFromToc}
  \tableofcontents
\end{KeepFromToc}
\mainmatter

\chapter{Phonology}

\section{Phonemic Inventory}
\begin{table}[ht]
  \centering
  \begin{tabular}{*{7}{c}}
    \toprule
              & Labial & Alveolar & Velar & Glottal \\\midrule
    Plosive   & p      & t        & k     & ʔ       \\
    Nasal     & m      & n        &       &         \\
    Fricative & ɸ      & s        & x     & h       \\
    Sonorant  & w      & l        &       &         \\
    \bottomrule
  \end{tabular}
  \caption{Consonant Inventory}
  \label{table:consonants}
\end{table}

%TODO: Add vowel realization blobs

\begin{figure}[ht]
  \centering
  \begin{vowel}
    \vpoint{0}{0}{i}
    \vpoint{0}{2}{u}
    \vpoint{1.5}{1}{ə}
    \vpoint{3}{1}{a}
    \varrow{a}{u}
    \varrow{a}{i}
  \end{vowel}
  \caption{Phonemic Vowels}
  \label{table:vowel_phonemes}
\end{figure}

\begin{figure}[ht]
  \centering
  \begin{vowel}
    \vpoint{0}{0}{i}
    \vpoint{0}{2}{u}
    \vpoint{1.5}{1}{ə}
    \vpoint{3}{1}{a}
  \end{vowel}
  \caption{Phonetic Vowel Ranges}
  \label{table:vowel_phones}
\end{figure}

\section{Phonotactics}\label{sec:phonotactics}

(C)V(V)
\begin{description}
  \item[C:] Any consonant
  \item[V:] Any vowel
\end{description}

Diphthongs are permitted, provided the vowels differ in height.
Under this rule, the diphthongs /iu/ and /ui/ are disallowed.
The similar sequences /iʔu/ and /uʔi/, however, are permitted.

\section{Stress}
\subsection{Timing}
\langname\ is considered syllable-timed, wherein all syllables are pronounced for approximately the same length of time.

\subsection{Placement}
Stress is typically placed on the first syllable in a word.
The only exception is when a word begins with /ə/, in which case it is placed on the second syllable.
Stressed /ə/ is phonologically unstable, which makes it susceptible to shifting elsewhere.
This can be observed in the following hypothetical forms: /ə.pa.lu/ [əˈpalu], and /pa.lu/ [ˈpalu].
Note how both forms are stressed on /pa/, even though the syllable is positioned differently in the word.

Additionally, a small number of morphemes are never able to be stressed, and so stress will shift based on the presence of these morphemes as well.

\subsection{Phonetic Realization}

\section{Allophony} %TODO: Add more to this section

/h/ realized as [ɣ] in stressed syllables

\phonc{h}{ɣ}{\phold V\phonfeat{+stress}}

/ɘ/ is realized as [e] in stressed syllables

\phonc{ə}{e}{\phold\phonfeat{+stress}}

/ɸ/ and /h/ are in free variation before /u/

/au/ monophthongizes to [o] after glottal consonants

\phonc{au}{o}{C\phonfeat{+glottal}\phold}


\part{Morphology}
\chapter{Nominal}
\section{Alignment}

\langname\ nouns are quite analytic.
A typical noun consists of a stem, plus an affix denoting case, and optional plural marking.
This may be either ergative or absolutive, though the absolutive is unmarked.
The following points give an indication of when each case should be used.
\begin{itemize}
  \item The agent of a transitive verb (A) is marked with ergative case
  \item The core argument of an intransitive verb (S) and the patient of a transitive verb (P) are both marked with absolutive case.
\end{itemize}

\pex<alignment>
\a<itrns>
\begingl
\glpreamble hawi keke
\pronounced{ˈɣa.wi ˈke.kə}\endpreamble
Ø-hawi[\textsc{abs-}rabbit]
keke[\textsc{npst.ipfv-}eat]
\glft `The rabbit is eating'
\endgl

\a<trns>
\begingl
\glpreamble hahawi keke ɸumau
\pronounced{ˈɣa.ɣa.wi ˈke.kə ˈɸu.mau}\endpreamble
ha\textasciitilde hawi[\textsc{erg\textasciitilde}rabbit]
keke[\textsc{npst.ipfv-}eat]
ɸumau[\textsc{abs,}grass]
\glft `The rabbit is eating the grass'
\endgl
\xe

Note that the stressed allophone of /h/ remains when reduplicated, this will be further explored in the following sections.

Notice that ``rabbit'' is declined in a different case for these two similar sentences.
The ergative is marked through reduplication of the first syllable.

\section{Plurality}

Plurals are formed with an affix \langword{me-}.
To illustrate its use we can revisit Example \getfullref{alignment.trns}.
`\langword{hahawi keke ɸumau}'.
If we want to pluralize \langword{hahawi,} we may expect \langword{mehahawi.} This is not the case however.
\langword{-me} is inserted between the root and the reduplicated ergative marking.
The reduplicated segment is not changed though, so the correct plural is \langword{hamehawi.} \langword{me-} is most accurately described as an interfix.

\begin{paracol}{2}
\pex<vowel-initial>
\a
\begingl
\glpreamble ana
\pronounced{ˈa.na}\endpreamble
Ø-ana[\textsc{abs-}eye]
\endgl
\switchcolumn

\a<epenthesis>
\begingl
\glpreamble aʔana
\pronounced{ˈa.ʔa.na}\endpreamble
aʔ\textasciitilde ana[\textsc{erg\textasciitilde }eye]
\endgl
\switchcolumn

\a
\begingl
\glpreamble meana
\pronounced{ˈme.a.na}\endpreamble
<me>Ø-ana[\textsc{<pl>abs-}eye]
\endgl
\switchcolumn

\a
\begingl
\glpreamble ameana
\pronounced{ˈa.mə.a.na}\endpreamble
a<me>ana[\textsc{erg<pl>\textasciitilde}eye]
\endgl
\xe
\end{paracol}
\begin{paracol}{2}
\pex<consonant-initial>
\a
\begingl
\glpreamble hawi
\pronounced{ˈɣa.wi}\endpreamble
Ø-hawi[\textsc{abs-}rabbit]
\endgl
\switchcolumn

\a
\begingl
\glpreamble hahawi
\pronounced{ˈɣa.ɣa.wi}\endpreamble
ha\textasciitilde hawi[\textsc{erg\textasciitilde }rabbit]
\endgl
\switchcolumn

\a
\begingl
\glpreamble mehawi
\pronounced{ˈme.ha.wi}\endpreamble
<me>Ø-hawi[\textsc{<pl>abs-}rabbit]
\endgl
\switchcolumn

\a
\begingl
\glpreamble hamehawi
\pronounced{ˈɣa.mə.ɣa.wi}\endpreamble
ha<me>hawi[\textsc{erg<pl>\textasciitilde}rabbit]
\endgl
\xe
\end{paracol}

Note how many forms of \langword{hawi} maintain [ɣ] even in unstressed envirenments.
This is because the reduplication causes it to be preserved.

In Example \getfullref{vowel-initial.epenthesis}, an epenthetic /ʔ/ has been inserted.
This is done because of restrictions surrounding diphthongs.
The full explanation can be found in \Sref{sec:phonotactics}.

For stems beginning with a syllable containing a diphthong, the reduplication surfaces a bit differently:

\begin{paracol}{2}
\pex
\a
\begingl
\glpreamble auna
\pronounced{ˈau.na}\endpreamble
Ø-auna[\textsc{abs-}moon]
\endgl
\switchcolumn

\a
\begingl
\glpreamble aʔauna
\pronounced{ˈa.ʔo.na}\endpreamble
aʔ\textasciitilde auna[\textsc{erg\textasciitilde}moon]
\endgl
\switchcolumn

\a
\begingl
\glpreamble meauna
\pronounced{ˈme.au.na}\endpreamble
<me>Ø-auna[\textsc{<pl>abs-}moon]
\endgl
\switchcolumn

\a
\begingl
\glpreamble ameauna
\pronounced{ˈa.mə.au.na}\endpreamble
a<me>auna[\textsc{erg<pl>\textasciitilde}moon]
\endgl
\xe
\end{paracol}
%TODO: Write a CVV-initial inflection example

\section{Indefinite Constructions}
\subsection{Indicative Mood}
Indefinite pronouns in particular don't exist as a separate class of words in \langname .
Rather, these sorts of constructions are created with interrogatives, and in certain situations leverage uncertainty evidentials, to impart the same meaning.

When responding to something where the answer would typically be an indefinite pronoun, a few strategies can be taken.
\subsubsection{Response}
Firstly, and often preferred, is the method of asking a question in return.
For example, if one were to ask ``Who touched that?'', the most simple and standard reply would be ``Who did?''.
In English this may be seen as sarcastic and possibly rude, but this is not the case with \langname .

If one were to reference \Sref{sec:certainty}, they may be wondering why these certainty particles are not more widely used in these scenarios.
The usages of these particles refers more to having an understanding (rather than strict knowing), and thus can only be used in certain scenarios.
More details about these particles in particular will be found in the corresponding section listed above.

This follows for any sort of nonpolar question.

The example exchange noted above has been reproduced in a glossed form to illustrate this pattern.

\pex<indef-pronominals>
\a<question>
\begingl
\glpreamble papane mutaʔali
\pronounced{ˈpa.pa.nə ˈmu.ta.ʔa.li}\endpreamble
pa\textasciitilde pane[erg\textasciitilde person]
mutaʔali[\textsc{pst.pfv:}touch\textsc{[q]}]
\glft `Who touched (that)?'
\endgl

\a<answer>
\begingl
\glpreamble pane
\pronounced{ˈpa.nə}\endpreamble
pane[person]
\glft `Someone'
\endgl
\xe

This short exchange actually illustrates several concepts simultaneously, many of which will be further addressed in later sections.

Firstly, in Example \getfullref{indef-pronominals.answer}, an utterance which would be typically translated as a simple interrogative `who?' is treated as an indefinite pronoun, which is an entirely grammatical response to this question.
This is also significant in that it is not a response to the question meant as a repair strategy.
These questions are constructed entirely different, and are detailed in \Sref{sec:repair_response}.

In regards to Example \getfullref{indef-pronominals.question}, it should be noted that demonstratives are often omitted for transitive verbs with 3\textsuperscript{rd} person agents, and the agent is inflected ergatively instead.

\subsubsection{Indefinite Agent}
If the agent of a verb would typically be an indefinite pronoun, a combination of peripherasis With additional verbs, and uncertainty evidentials is used.

\ex
\begingl
\glpreamble pane wu ehaixa i munihasi lu me malaku, me ɸai ʔike
\pronounced{ˈpa.ne ˈwu ˈe.hai.xa ˈi ˈmu.ni.ha.si ˈlu ˈme ˈma.la.ku | ˈme ˈɸai ˈʔi.kə}\endpreamble
\nogloss{\lbrack}
pa\textasciitilde pane[\textsc{erg\textasciitilde}person]
wu[on\_top]
ehaixa[\textsc{name}]
i[inside]
munihasi[\textsc{pst.pfv}:give\textsc{:inv}]
lu[\textsc{obv}]
me[\textsc{1sg}]
malaku[cat]
\nogloss{\rbrack}
me\textasciitilde me[\textsc{erg\textasciitilde 1sg}]
ɸai[\textsc{neg}]
ʔike[know]
\glft `(Someone) from Ehaixa gave me this cat, I don't know (them).'
\endgl
\xe

In this case, the construction with \langword{ʔike} is necessary to specify the exact agent is unknown.
Evidentials cannot be used here, because it is not the giving itself which is uncertain, and the evidentials modify verbs.

In this scenario, the verb phrase could also be placed into a relative clause, although this is often not particularly practical unless paired with \nameref{sec:ellipsis}.
\subsubsection{Indefinite Patient}
\subsection{Interrogative Mood}
Interrogative contexts are one of such situations in which the certainty evidentials will be utilized most often.
In these situations the evidentials are appropriate, because the speaker of the utterance is expressing uncertainty about whether an event has happened in regards to a particular person.

\ex
\begingl
\glpreamble tesune mukula pane
\pronounced{ˈte.su.nə ˈmu.kula ˈpa.nə}\endpreamble
tesune[\textsc{uncert}]
mukula[\textsc{pst.pfv:}hurt\textsc{[q]}]
pane[person]
\glft `Did you hurt someone?'
\endgl
\xe

In this sort of situation, evidentials are preferred.
This has to do with specifically which part of the utterance the speaker is unsure of.
Here, the speaker is unsure of whether the event (hurting) actually occurred or not, thus a construction which modifies the verb is sensible.
\langword{Pane} is simply here as a way of marking the 3\textsuperscript{rd} person, because if the object was completely omitted, a direct verb in this configuration would have a 1\textsuperscript{st} person object.
\chapter{Adjectival}\label{ch:adjectives}
Adjectives are not a unique class of words in \langname . What may look like ``adjectives'' on the surface are simply nouns.

\ex
\begingl
\glpreamble masaʔe si sixi
\pronounced{ˈma.sa.ʔə ˈsi ˈsi.xi}\endpreamble
\nogloss{\lbrack}
masa[sun]
ʔe[\textsc{assoc}]
si[brightness]
\nogloss{\rbrack}
sixi[\textsc{npst.ipfv}-shine]
\glft `The bright sun shines.'
\endgl
\xe

If the noun being modified in this way has ergative marking, it should be noted that the noun \textit{does not} inflect in agreement with the main noun.

\section{Adjective Ordering}

\langname 's\ basic adjective ordering is: «opinion» «size» «physical quality» «shape» «age» «color» «origin» «material» «type» «purpose»

In some cases however, this basic ordering may be deviated from. A single adjective may be placed before \langword{ʔe,} allowing the main noun itself to be dropped, and the main to be referenced in futher discourse using the promoted adjective + ʔe as a logophoric pronoun. This is particularly useful when many of the same object with similar but differing qualities are being discussed for extended lengths of time (for example, a discussion about two different people, or about several types of a similar object). It may also be used, as seen below to chain clauses together.

\ex
\begingl
\glpreamble masa siʔe sixi, siʔe kaʔa
\pronounced{ˈma.sa ˈsi.ʔe ˈsi.xi | ˈsi.ʔe ˈka.ʔa}\endpreamble
\nogloss{\lbrack}
masa[\textsc{sun}]
si[brightness]
ʔe[\textsc{assoc}]
\nogloss{\rbrack}
sixi[\textsc{npst.ipfv}-shine]
si-ʔe[brightness-ʔe]
kaʔa[happiness]
\glft `The bright sun shines, and its light makes me happy.'
\endgl
\xe

%TODO: Cover examples with many adjectives 
\chapter{Verbal}

Morphologically, verbs are quite simple. Tense is split into past and non-past forms, while aspect is split simply along perfectivity. This results in a simple set of four possible inflections, three of which are overtly marked.

\begin{table}[ht]
  \centering
  \begin{tabular}{*{3}{c}}
    \toprule
                 & Nonpast & Past  \\\midrule
    Imperfective & ∅-      & i-    \\
    Perfective   & V-      & m(u)- \\
    \bottomrule
  \end{tabular}
  \caption{Verb Inflection}
  \label{table:verb-inflection}
\end{table}

The \textsc{npst.pfv} form can be described as \morphtext{V,}, where V is a copy vowel dependent on the adjacent syllable. For this reason, i-stem verbs will display a minimal pair between initial- and second-syllable stress in the \textsc{npst.pfv} and \textsc{pst.ipfv} forms.

When a verb stem begins in a vowel, epenthetic \phontext{ʔ} will be inserted for both the \textsc{npst.pfv} and \textsc{pst.ipfv} forms.

\part{Syntax}

\chapter{Nominal}

\section{Adpositions}

\begin{description}
  \item[\langword{wu}:] On top (of), from above
  \item[\langword{he}:] Under, from below
  \item[\langword{sa}:] Outside of, from outside
  \item[\langword{i}:] In, from inside
  \item[\langword{ana}:] Next to, near
\end{description}

\ex
\begingl
meme[\textsc{1sg:erg}]
sulau[shoo]
mesikima[\textsc{pl:}fly]
he[under]
line[light]
\glft `I am shooing the flies out from under the light'
\endgl
\xe

This example, in regards to the preposition itself, is has ambiguous directionality, in that the flies may be shooed either towards or away from the light. Generally this must be inferred, explicit marking of this is uncommon. In the case of \langword{sulau} however, the latter meaning is the only sensible reading. A similar but opposite effect occurs with \langword{kamai} `to send', which is unabiguously read with the former meaning, illustrated below:

\ex
\begingl
\glpreamble mamemalaku ikamai ta he sesiti
\pronounced{ˈma.mə.ma.la.ku ˈi.ka.mai ˈta ˈɣe ˈse.si.ti}\endpreamble
mamemaleku[\textsc{erg:pl:}cat]
ikamai[\textsc{pst.pfv:}send]
ta[\textsc{prox}]
he[under]
sesiti[blanket]
\glft `The cats hid under the blanket'
\endgl
\xe

The verb \langword{kamai} is strictly transitive, so \langword{ta} can be used resumptively to form what is essentially a reflexive construction.
\chapter{Verbal}

\section{Focus}

\ex
\begingl
\glpreamble ɸumau keke hahawi
\pronounced{ˈɸu.mau ˈke.kə ˈɣa.ha.wi}\endpreamble
Ø-ɸumau[\textsc{abs-}grass]
Ø-keke[\textsc{npst.ipfv-}eat]
ha\textasciitilde hawi[\textsc{erg\textasciitilde}rabbit]
\glft `The grass is being eaten by the rabbit'
\endgl
\xe

Because the grass is still the patient of the verb, it is still marked with the ergative. Fronted arguments of transitive verbs become focused. A passive construction will be used in translation to English. This is solely to approximate the topicalization, as this example is not a true passive (the verb's valency is not decreased). Arguments in default position can be focused, albeit in a different manner. Returning to Example \getfullref{alignment.trns}, but with the agent explicitly focused:

\ex
\begingl
\glpreamble hahawi keke lu ɸumau
\pronounced{ˈɣa.ha.wi ˈke.kə ˈlu ˈɸu.mau}\endpreamble
ha\textasciitilde hawi[\textsc{erg\textasciitilde}rabbit]
keke[\textsc{npst.ipfv-}eat]
lu[\textsc{obv}]
ɸumau[\textsc{abs,}grass]
\glft `The rabbit (as opposed to something else) is eating the grass'
\endgl
\xe

By marking the already established patient with the obviative\footnotemark, it puts more focus on the (unmarked) proximal argument than would be typical.

\footnotetext{The standard use of proximate/obviate morphology is falling out of use in favor of case marking, Remaining instances have either become fossilized in expressions and idioms, or fulfilled another grammatical purpose, as seen here.}

\section{Labile Verbs}\label{sec:labile_verbs}

A labile verb is a verb that can be either transitive or intransitive, and whose subject when intransitive corresponds to its direct object when transitive. They are also sometimes referred to as ``S=O ambitransitive`` verbs. A prototypical example of this being ``John tripped'' in contrast with ``John tripped Tim''. Unlike a typical ambitransitive verb, the subject's role changes.

\ex
\begingl
\glpreamble ɸihaʔau me
\pronounced{ˈɸi.ha.ʔo ˈme}\endpreamble
ɸihaʔau-∅[trip\textsc{-dir}]
me[\textsc{1sg.abs}]
\glft `You tripped me'
\endgl
\xe

\ex
\begingl
\glpreamble me ɸiahau
\pronounced{ˈme ˈɸi.a.ho}\endpreamble
me[\textsc{1sg.abs}]
ɸiahau-∅[trip-\textsc{-dir}]
\glft `(You) tripped me'
\endgl
\xe

These first two examples utilize concepts which have previously been covered in \Sref{sec:person_hierarchy}. The following utilizes the proximate particle, \langword{ta,} in order to mark \langword{me} as the most agentlike argument of a transitive verb. As such, there is no possibility of inferring an agent of \langword{ɸihaʔau.} In this way, labile verbs can be expressed without the need for a dummy agent.

\ex
\begingl
\glpreamble ta me ɸihaʔau
\pronounced{ˈta ˈme ˈɸi.ha.ʔo}\endpreamble
ta[\textsc{prox}]
me[\textsc{1sg.abs}]
ɸihaʔau-∅[trip\textsc{-dir}]
\glft `I tripped'
\endgl
\xe


\chapter{Discourse}

\section{Repair Questions}
As an example, lets imagine a hypothetical speaker just said the following:
\ex
\begingl
\glpreamble hahawi keke lu ɸumau
\pronounced{ˈɣa.ɣa.wi ˈke.kə ˈlu ˈɸu.mau}\endpreamble
ha\textasciitilde hawi[\textsc{erg\textasciitilde}rabbit]
keke[\textsc{npst.ipfv-}eat]
lu[\textsc{obv}]
ɸumau[\textsc{abs,}grass]
\glft `The rabbit (as opposed to something else) is eating the grass.'
\endgl
\xe

\subsection{Nominals}

If someone mishears, or for whatever reason needs clarification on the arguments of a transitive verb, the obviative and proximate markers can be used.

\begin{paracol}{2}
If the listener only hears \langword{hahawi ke\-ke lu}, and not the \textit{patient}, the listener can ask the following:
\ex
\begingl
\glpreamble lu?
\pronounced{ˈlu}\endpreamble
lu[\textsc{obv}]
\glft `Eating what?'
\endgl
\xe
\switchcolumn

If a listener only hears \langword{keke lu ɸumau,} and not the \textit{agent}, the listener can ask the following:

\ex
\begingl
\glpreamble ta?
\pronounced{ˈta}\endpreamble
ta[\textsc{prox}]
\glft `What is eating grass?'
\endgl
\xe
\end{paracol}

\subsection{Verbal}
If our listener only heard \langword{``hahawi --- lu ɸumau'',} the response may be:
\ex
\begingl
\glpreamble ta sihu lu?
\pronounced{ˈta ˈsi.hu ˈlu}\endpreamble
ta[\textsc{prox}]
sihu[happen]
lu[\textsc{obv}]
\glft `The rabbit is doing what to grass?'
\endgl
\xe

\langword{ta} and \langword{lu} are used here in a resumptive fashion, rather than repeating the content words. This implies more confidence, in that repeating \langword{hahawi} or \langword{ɸumau} may imply that the listener is also unsure of these components as well, rather than just the verb.

Because this phrase is somewhat of a standard one, it is shortened in colloquial speech. The most aggresive of these shortenings being [ˈtasul(ə)].

\section{Responding to Repair Questions}\label{sec:repair_response}

Repair questions can be responded to quite similarly to how a ``standard'' question would be. The main difference is the necessity of the associative particle, \langword{ʔe} to connect \langword{ta} or \langword{lu} to the appropriate content word(s). For example:

\ex
\begingl
\glpreamble ta ʔehawi
\pronounced{ˈta ˈʔe.ha.wi}\endpreamble
ta[\textsc{prox}]
ʔe[\textsc{assoc}]
hawi[rabbit]
\glft `The rabbit (is eating grass).'
\endgl
\xe
\section{Self-Correction}
\subsection{In conjunction with \langword{ʔe} Ellipsis}

The same structures used to respond to repair questions may be employed to correct or clarify the ellipsed NP, for example if the remaining information is too ambiguous, incorrect, or is simply no longer relevant.

\ex<adjective_correction>
\begingl
\glpreamble tatakaʔe ala ta ʔekatu kulasi me
\pronounced{ˈta.ta.ka.ʔə ˈa.la ˈta ˈʔe.ka.tu ˈku.la.si ˈme}\endpreamble
tataka[\textsc{erg:}rock]
ʔe[\textsc{assoc}]
ala[white]
ta[\textsc{prox}]
ʔe-katu[ʔe-sharpness]
kulasi[hurt\textsc{:inv}]
me[\textsc{abs:1sg}]
\glft `The white --- no, sharp --- rock is hurting me.'
\endgl
\xe

When \langword{ʔe} is used in discourse repair, it attaches to the correction, rather than to \langword{ta} or \langword{lu.} This the the opposite of when \langword{ʔe} is used associatively (see \Sref{ch:adjectives}), where \langword{ʔe} attaches to the noun being described.

%TODO: Self-repaired nouns and verbs

\subsection{Noun Repair}
\ex<adjective_correction>
\begingl
\glpreamble hahawi ta ʔetatakaʔe ala kulasi me
\pronounced{ˈɣa.ha.wi ta ˈʔe.ta.ta.ka.ʔə ˈa.la ˈku.la.si ˈme}\endpreamble
hahawi[\textsc{erg:}rabbit]
ta[\textsc{prox}]
ʔe[\textsc{assoc}]
tataka[\textsc{erg:}rock]
ʔe[\textsc{assoc}]
ala[white]
kulasi[hurt\textsc{:inv}]
me[\textsc{abs:1sg}]
\glft `The white rabbit --- no, rock ---  is hurting me.'
\endgl
\xe

The patient can be repaired the same way, though using \langword{lu} instead of \langword{ta.}
\subsection{New Adjectives}
Similarly, if a noun which previously had no adjectives modifying it needs to have one added in the middle of discourse, the following can be done:

\ex
\begingl
\glpreamble ta nene xase lu ʔe kela
\pronounced{ˈta ˈne.nə ˈxa.sə ˈlu.ʔə ˈke.la}\endpreamble
ta[\textsc{prox}]
nene[paint]
xase[spill]
lu[\textsc{obv}]
ʔe[\textsc{assoc}]
kela[green]
\glft `The paint spilled, it's green.'
\endgl
\xe

%? Does green exist as its own basic term? Decide.

Special attention should be given when referencing arguments of labile verbs (see \Sref{sec:labile_verbs}). While \langword{ta} marks \langword{nene} as the most agent-like, it is still morphologically and syntactically the verb's patient (note the lack of ergative marking). Because of this, core (S) arguments of labile verbs should be used with the resumptive pronoun used with patients, \langword{lu.}

Also, adjectival additions such as this are strongly preferred to be done \textit{after} the verb, particularly with intransitive verbs. These types of corrections are done differently in that the adjective \textit{is} associated to \langword{lu}. This is different from correcting the use of an incorrect adjective, seen in Example \getfullref{adjective_correction}.

\section{Affirmative/Negative Particles}

\langname\ has an extensive system of discouse particles, particularly those relating to certainty, and various levels of agreement. In most senarios, these are simply placed before the verb phrase, but a select few have special clitic forms, which attach directly to the verb.

\subsection{Certainty}

\langname 's `certainty particles', as they will be referred to from now on, exist on a continuum, ranging from \langword{naʔu} `unwavering certainty of falsehood' to \langword{tamuka} `unwavering certainty of truthfullness', with medians such as \langword{tesune} `firm skepticism', and \langword{ewili} `easily swayed uncertainty'.

A more comprehensive list is as follows: \langword{naʔu, lelewine, tesune, ewili, kisuɸi, mili, tamuka.} The undescribed forms fall somewhere within the space in the continuum, such that their definitions are solely defined by their relationships to each other.
\subsubsection{\langword{Lelewine}}
For example, if a businessperson of questionable legitimacy approaches you with a suspiciously low price on an expensive item, one could reply using these particles to show (dis)trust.

\ex
\begingl
\glpreamble lelewine i naʔa lu me memalaku ilaʔe
\pronounced{ˈle.lə.wi.nə ˈi ˈna.ʔa ˈlu ˈme ˈme.ma.la.ku ˈi.la.ʔə}\endpreamble
lelewine[\textsc{uncert}]
i[inside]
naʔa[\textsc{npst.ipfv:}sell]
lu[\textsc{obv}]
me[\textsc{1sg}]
memalaku[pl:cat]
ila\footnotemark[wealth]
ʔe[\textsc{assoc}]
\glft `You are selling me cats, for only this much (surely this isn't right, but just maybe)?'
\endgl
\xe

Interrogative meaning is additionally conveyed prosodically, which is particularly important in that ditransitives have fixed V2 order.

\footnotetext{`Wealthy cats' are not particularly sensible, and therefore this associative is used to refer to the cost of them instead. This happens often when an adjective is not prototypically used to describe something -- its meaning in this context shifts slightly.}

\ex<uncertain-clitic>
\begingl
\glpreamble i ewi-naʔa lu me memalaku ilaʔe
\pronounced{ˈi ˈe.wi.na.ʔa ˈlu ˈme ˈme.ma.la.ku ˈi.la.ʔə}\endpreamble
i[inside]
ewi-naʔa[\textsc{uncert-npst.ipfv:}sell]
lu[\textsc{obv}]
me[\textsc{1sg}]
memalaku[pl:cat]
ila[wealth]
ʔe[\textsc{assoc}]
\glft `You are selling me cats, for only this much (surely this isn't right, but just maybe)?'
\endgl
\xe

The particle \langword{lelewine} can be shortened and attached (orthographically as well) as in the above, to have the same meaning, but trading a small amount of clarity for a stylistic choice of words. The shortened, attached forms are also preferred (if they exist), when a verb is used on it's own as another form of echo question.

% \begin{tikzpicture}
%   \contour[contour raise=0.5cm]
%   {|[10]Where |[3]are you |[6]go|[1]ing?|[0]}
% \end{tikzpicture}

\ex
\begingl
\glpreamble imipi ɸai luʔe paki?
\pronounced{ˈi.mi.pi ˈɸai ˈlu.ʔə ˈpa.ki}\endpreamble
imipi[\textsc{pst.ipfv}:think]
ɸai[\textsc{neg}]
lu[\textsc{obv}]
ʔe[\textsc{assoc}]
paki[truth]
\glft `You weren't thinking of me truthfully?'
\endgl
\xe

Note the use of the imperfective here, this is important because it shows the speaker still knows they have not swayed the other person. To use the perfective would, in this senario, give the impression of an implied `but now you have', which is not the case here.

In response to Example \getfullref{uncertain-clitic}, a still uncertain buyer may affirm previous thoughts with one of the following:

\ex
\begingl
\glpreamble ewi-naʔa
\pronounced{ə.wiˈna.ʔa}\endpreamble
i[inside]
ewi-naʔa[\textsc{uncert-npst.ipfv:}sell]
lu[\textsc{obv}]
me[\textsc{1sg}]
memalaku[pl:cat]
ila[wealth]
ʔe[\textsc{assoc}]
\glft `You're selling? I don't know about this...'
\endgl
\xe

Alternatively, one can simply repeat (or choose another) certainty particle to use in isolatio. When used this way, the full form is mandatory.

\ex
\begingl
\glpreamble lelewine...
\pronounced{ ˈle.lə.wi.nə}\endpreamble
lelewine[\textsc{uncert}]
\glft `I'm not so sure about this...'
\endgl
\xe

%? This file will be used to condense examples into a sort of grammatical phrasebook, but it is very much WIP
% \part{Example Sentences}
\ex
\begingl
masa[sun]
«shine»[\textsc{npst.ipfv}-shine]
\glft `The sun shines.'
\endgl
\xe

\ex
\begingl
masa[sun]
pa[now]
«shine»[\textsc{npst.ipfv}-shine]
\glft `The sun is shining.'
\endgl
\xe

\ex
\begingl
masa[sun]
i-«shine»[\textsc{pst.ipfv}-shine]
\glft `The sun shone.'
\endgl
\xe

\pex
\a
\begingl
masa[sun]
«shine»[\textsc{npst.ipfv}-shine]
\glft `The sun will shine.'
\endgl
\a
\begingl
masa[sun]
i-ti[\textsc{pst.ipfv-}shine,]
alete[thus]
∅-ti[\textsc{npst.ipfv-}shine]
\glft `The sun will shine.'
\endgl
\xe

\pex
\a
\beginglpanel
\glpreamble
\endpreamble
masa[sun]
pa[now]
«shine»[\textsc{npst.ipfv}-shine]
\endgl
This usage of the imperfective typically requires a true duration, and as such this construction as written may be confused for a standard progressive reading, as seen above.
\endpanel
`The sun has been shining.'

\a
\beginglpanel
\glpreamble
\endpreamble
masa[sun]
neʔe[hour]
«shine»[\textsc{npst.ipfv}-shine]
\endgl
This usage of the imperfective will be much more commonly encountered.
\endpanel
`The sun has been shining for one hour.'
\xe

\ex
\begingl
masa[sun]
«tomorrow»[tomorrow]
«shine»[\textsc{npst.ipfv}-shine]
\glft `The sun will shine tomorrow.'
\endgl
\xe

% The sun shines brightly.


%? adj are nouns in relclauses, can put large NPs here also

%* Find a proper place for this
\section{Adpositional Stuff}

\begin{description}
  \item[\langword{wu}:] On top (of), from above
  \item[\langword{he}:] Under, from below
  \item[\langword{sa}:] Outside of, from outside
  \item[\langword{i}:] In, from inside
  \item[\langword{ana}:] Next to, near
\end{description}

\ex
\begingl
meme[\textsc{1sg:erg}]
sulau[shoo]
mesikima[\textsc{pl:}fly]
he[under]
line[light]
\glft `I am shooing the flies out from under the light'
\endgl
\xe

This example, in regards to the preposition itself, is has ambiguous directionality, in that the flies may be shooed either towards or away from the light. Generally this must be inferred, explicit marking of this is uncommon. In the case of \langword{sulau} however, the latter meaning is the only sensible reading. A similar but opposite effect occurs with \langword{kamai} `to send', which is unabiguously read with the former meaning, illustrated below:

\ex
\begingl
\glpreamble mamemalaku ikamai ta he sesiti
\pronounced{ˈma.mə.ma.la.ku ˈi.ka.mai ˈta ˈɣe ˈse.si.ti}\endpreamble
mamemaleku[\textsc{erg:pl:}cat]
ikamai[\textsc{pst.pfv:}send]
ta[\textsc{prox}]
he[under]
sesiti[blanket]
\glft `The cats hid under the blanket'
\endgl
\xe

The verb \langword{kamai} is strictly
\backmatter
% bibliography, glossary and index would go here.

\end{document}
